\usepackage{etoolbox}
\usepackage[utf8]{inputenc}

\usepackage{mathtools}
\usepackage{amsmath}
\usepackage{pgfplots}
\usepackage{dsfont}
\usepackage{faktor}
\usepackage{ latexsym }
\usepackage{amssymb }
\usepackage{xcolor}
\usepackage{colortbl}
\usepackage{array}

\usepackage{fontawesome5}
\usepackage{MnSymbol,wasysym} %Pour des smileys

\usepackage{multirow,bigdelim}
\usepackage{multicol}

\usepackage[framemethod=default]{mdframed}
\usepackage{pifont}

\newcolumntype{C}[1]{>{\centering\let\newline\\ \arraybackslash\hspace{0pt}}m{#1}}

\usefonttheme[onlymath]{serif} %De belles formules
\usepackage{xcolor}


\newcommand\F{\mathbb{F}}
\newcommand\Fq{\mathbb{F}_q}
\newcommand\Fqm{\mathbb{F}_{q^m}}
\newcommand\Z{\mathbb{Z}}
\newcommand\A{\mathbb{A}}
\newcommand\PP{\mathbb{P}}
\newcommand\N{\mathbb{N}}
\newcommand\T{\mathbb{T}}
\newcommand\R{\mathbb{R}}

\newcommand{\calC}{\mathcal{C}}
\newcommand{\calD}{\mathcal{D}}
\newcommand{\calF}{\mathcal{F}}
\newcommand{\calH}{\mathcal{H}}
\newcommand{\calL}{\mathcal{L}}
\newcommand{\calP}{\mathcal{P}}
\newcommand{\calR}{\mathcal{R}}
\newcommand{\calX}{\mathcal{X}}
\newcommand{\calY}{\mathcal{Y}}

\newcommand\bfc{\vec{c}}
\newcommand\bfd{\vec{d}}
\newcommand\bfm{\vec{m}}
\newcommand\bfx{\vec{x}}
\newcommand\bfy{\vec{y}}
\newcommand\bfz{\vec{z}}


\newcommand\RS{\mathsf{RS}}
\newcommand\GRS{\mathsf{GRS}}
\newcommand{\set}[1]{\left\{#1\right\}}

\renewcommand{\vec}[1]{\boldsymbol{#1}}

\newcommand{\ps}[2]{\left\langle #1,#2 \right\rangle}
\renewcommand{\epsilon}{\varepsilon}

\DeclareMathOperator{\ev}{ev}
\DeclareMathOperator{\Span}{Span}
\DeclareMathOperator{\Supp}{Supp}
\DeclareMathOperator{\Tr}{Tr}
\newcommand{\degab}[1]{\deg_{a,b}\left(#1\right)}


% Bien aligner ses incides multi-lignes

\makeatletter
\newcommand{\subalign}[1]{%
	\vcenter{%
		\Let@ \restore@math@cr \default@tag
		\baselineskip\fontdimen10 \scriptfont\tw@
		\advance\baselineskip\fontdimen12 \scriptfont\tw@
		\lineskip\thr@@\fontdimen8 \scriptfont\thr@@
		\lineskiplimit\lineskip
		\ialign{\hfil$\m@th\scriptstyle##$&$\m@th\scriptstyle{}##$\hfil\crcr
			#1\crcr
		}%
	}%
}
\makeatother


\mdfdefinestyle{mdf}{innerleftmargin=0.3em,innerrightmargin=0.3em,innertopmargin=0.3em,innerbottommargin=0.3em,linecolor=white}





%-----Commandes Beamer %-----------------------

\DeclareOptionBeamer{compress}{\beamer@compresstrue}
\ProcessOptionsBeamer

\mode<presentation>

\useoutertheme[footline=authortitle, footline=frame number,subsection=false]{miniframes}
\useinnertheme{circles}
\usecolortheme{whale}
\usecolortheme{orchid}

\definecolor{beamer@blendedblue}{rgb}{0.137,0.466,0.741}

\setbeamercolor{structure}{fg=black}
\setbeamercolor{titlelike}{parent=structure}
\setbeamercolor{frametitle}{bg=structure!70,fg=white}
\setbeamerfont{frametitle}{size=\small,series=\bfseries}
\setbeamercolor{title}{fg=black}
\setbeamercolor{item}{fg=black}
\setbeamertemplate{page number in head/foot}[totalframenumber] %Pour ajouter le numéro des slides
\setbeamerfont{block title}{size=\small,series=\bfseries}

\setbeamercolor{block title example}{fg=white,bg=teal}
\setbeamercolor{block body example}{fg=black,bg=teal!10}


\setbeamercolor{block title alerted}{fg=white,bg=red!45!yellow}
\setbeamercolor{block body alerted}{fg=black,bg=red!40!yellow!10}


\setbeamertemplate{frametitle}{%
	\nointerlineskip%
	\begin{beamercolorbox}[wd=\paperwidth,ht=2.0ex,dp=0.6ex]{frametitle}
		\hspace*{1ex}\insertframetitle%
	\end{beamercolorbox}%
}

%Pour mettre les miniframes (ronds) sur la même ligne que ls sections
\makeatletter
\patchcmd{\slideentry}{\advance\beamer@tempdim by -.05cm}{\advance\beamer@tempdim by\beamer@vboxoffset\advance\beamer@tempdim by\beamer@boxsize\advance\beamer@tempdim by 1.2\pgflinewidth}{}{}
\patchcmd{\slideentry}{\kern\beamer@tempdim}{\advance\beamer@tempdim by 2pt\advance\beamer@tempdim by\wd\beamer@sectionbox\kern\beamer@tempdim}{}{}
\makeatother

\mode<all>
%\includeonlyframes{current}

\setbeamersize{text margin left=0.8cm,text margin right=0.8cm}

\setbeamertemplate{navigation symbols}{}  %Bye bye les boutons du bas

%----------------Espaces autour équations%---------

\makeatletter
\g@addto@macro\normalsize{%
	\setlength\belowdisplayskip{0.2em}
	\setlength\abovedisplayskip{0.2em}
}

%----------------


\vfuzz=25pt %Evite les alertes "Over full \vbox" jusqu'à 25pt.

%---------------- Tikz Packages ------------------
\usepackage{tikz}
\usetikzlibrary{tikzmark} %Permet de créer des noeuds dans du texte pour annoter dans un tikz ultérieur. Très pratique pour annoter les formules moches.
\usetikzlibrary{arrows,patterns,positioning,fit}
\usetikzlibrary{matrix,arrows,shapes,shapes.misc}
\usetikzlibrary{decorations.pathreplacing,angles,quotes}
\usetikzlibrary{decorations.markings,calc}
\usetikzlibrary{overlay-beamer-styles,backgrounds} %Pour utiliser alt dans tikz
\tikzset{>=stealth} % Change le style de flèche dans tikz

%--------- Un bout de code pour récupérer la couleur d'un noeud rectangulaire ou circulaire : noeud.f pour la couleur du fill, noeud.d pour celle du draw ---------------
% Voir https://tex.stackexchange.com/questions/602047/is-obtaining-the-color-of-a-node-possible

\makeatletter
\def\pgf@sh@fbg@circle{%
	\@ifundefinedcolor{pgffillcolor}{}{\xglobal\colorlet{\pgf@node@name.f}{pgffillcolor}}%
	\@ifundefinedcolor{pgfstrokecolor}{}{\xglobal\colorlet{\pgf@node@name.d}{pgfstrokecolor}}%
}
\def\pgf@sh@fbg@rectangle{%
	\@ifundefinedcolor{pgffillcolor}{}{\xglobal\colorlet{\pgf@node@name.f}{pgffillcolor}}%
	\@ifundefinedcolor{pgfstrokecolor}{}{\xglobal\colorlet{\pgf@node@name.d}{pgfstrokecolor}}%
}
\makeatother

%-Après avoir défini un \tikzmarknode{noeuf}{texte}, on crée un noeuf rectangulaire (aux coins arrondis) autour d'une certaine couleur et on annote avec une flèche.

\newcommand\highlightnode[2]{ %args: noeud, couleur
	\node (#1-frame)[rounded corners,fit=(#1),inner sep=2pt,fill=#2,fill opacity=0.17] {};
}

\newcommand\framenode[6][(0,0)]{ %args: (shift) noeud, position, couleur, texte , booléen flèche 0 = non / 1 = oui
	\highlightnode{#2}{#4}
	\node [#2-frame.f,#3 of = #2, shift={#1}](#2-text){#5};
	\ifx1#6 	\draw[#2-frame.f,<-] (#2-frame) -- (#2-text); \fi		
}

%-----------Couleurs-------------
\colorlet{bgreen}{green!60!black} 
\colorlet{bred}{red!70!black} 
\colorlet{alertcolor}{red!60!yellow}
\renewcommand{\alert}[1]{\textcolor{alertcolor}{#1}}
\newcommand{\new}[1]{\textcolor{blue!60!green}{#1}}
\newcommand{\blue}[1]{\textcolor{beamer@blendedblue!70!black}{#1}}
\newcommand{\rawblue}[1]{\textcolor{blue}{#1}}
\newcommand{\details}[1]{\scalebox{0.8}{\textcolor{structure!20!black!80}{#1}}}





