\documentclass[
10pt, % Set the default font size, options include: 8pt, 9pt, 10pt, 11pt, 12pt, 14pt, 17pt, 20pt
%
aspectratio=169, % Uncomment to set the aspect ratio to a 16:9 ratio which matches the aspect ratio of 1080p and 4K screens and projectors
]{beamer}

\usepackage{xcolor,colortbl}
\usepackage{booktabs}
\usetheme[progressbar=frametitle]{metropolis}
\usepackage{appendixnumberbeamer}

\usepackage{booktabs}
\usepackage[scale=2]{ccicons}

\usepackage{pgfplots}
\usepgfplotslibrary{dateplot}

\usepackage{xspace}
\newcommand{\themename}{\textbf{\textsc{metropolis}}\xspace}

\usepackage[utf8]{inputenc}
\usepackage{caption}
\usepackage{xcolor}
\usepackage{amsmath}
\usepackage{amssymb}
\usepackage{amsfonts}
\usepackage{verbatim}
\usepackage{amsthm}
\usepackage{geometry}
\usepackage{hyperref} %Références cliquables
\usepackage{cleveref}
\usepackage{csquotes}
\usepackage{mathtools}
\usepackage{relsize}
\usepackage{tcolorbox}
\usepackage{tikz}
\usepackage{fontawesome5}
\usetikzlibrary{snakes,arrows,shapes}
%%%%%%%%%%%%%%%%%%%%%%%%%%%%%%%%%%%%%%%%%%%%%%%%%%%%%%%%%
\definecolor{gold}{rgb}{0.83, 0.69, 0.22}
\definecolor{gamboge}{rgb}{0.89, 0.61, 0.06}
\definecolor{aliceblue}{rgb}{0.94, 0.97, 1.0}
\definecolor{ballblue}{rgb}{0.13, 0.67, 0.8}
\definecolor{charcoal}{rgb}{0.21, 0.27, 0.31}
\definecolor{lightkhaki}{rgb}{0.94, 0.9, 0.55}
\definecolor{applegreen}{rgb}{0.55, 0.71, 0.0}
\setbeamercolor{block title}{use=structure,fg=charcoal,bg=ballblue}
\setbeamercolor{block body}{use=structure,fg=charcoal,bg=aliceblue}

\setbeamercolor{progress bar}{fg=ballblue,bg=aliceblue}
\setbeamercolor*{structure}{bg=aliceblue,fg=ballblue}
\setbeamercolor{block title alerted}{fg=charcoal,bg=gamboge}
%body
\setbeamercolor{block body alerted}{fg= black,bg=lightkhaki}

%%%%%%%%%%%%%%%%%%%%%%%%%%%%%%%%%%%%%%%%%%%%%%%%%%%%%%%%%%5
\theoremstyle{plain}% default
\newtheorem{thm}{Theorem}[section]
\newtheorem{lem}[thm]{Lemma}
\newtheorem{prop}[thm]{Proposition}
\newtheorem*{cor}{Corollary}
\theoremstyle{definition}
\newtheorem{defn}{Definition}[section]
\newtheorem{exmp}{Example}[section]
\newtheorem{xca}[exmp]{Exercise}
\theoremstyle{remark}


%--------------- Commands

%------------Commands-----------%
\newcommand{\calA}{\mathcal{A}}
\newcommand{\calB}{\mathcal{B}}
\newcommand{\calS}{\mathcal{S}}
\newcommand{\calP}{\mathcal{P}}
\newcommand{\calH}{\mathcal{H}}
\newcommand{\calL}{\mathcal{L}}
\newcommand{\calC}{\mathcal{C}}
\newcommand{\calD}{\mathcal{D}}
\newcommand{\calO}{\mathcal{O}}
\newcommand{\calR}{\mathcal{R}}
\newcommand{\calT}{\mathcal{T}}
\newcommand{\calX}{\mathcal{X}}
\newcommand{\fqm}{\mathbb{F}_{q^m}}
\newcommand{\fq}{\mathbb{F}_{q}}
\newcommand{\fqo}{\mathbb{F}_{q_0^2}}
\newcommand{\F}{\mathbb{F}}
\newcommand{\Z}{\mathbb{Z}}
\newcommand{\PP}{\mathbb{P}}
\newcommand{\R}{\mathbb{R}}
\newcommand{\Tr}[1]{\operatorname{Tr}_{\mathbb{F}_{q^m}/\fq}\left(#1\right)}
\newcommand{\set}[1]{\left\{#1\right\}}
\newcommand{\Floor}[1]{\left\lfloor #1 \right\rfloor}
\newcommand{\Span}[1]{\operatorname{Span}\left\lbrace #1\right\rbrace }
\newcommand{\LT}[1]{\operatorname{LT}\left(#1\right)}
\newcommand{\Supp}{\operatorname{Supp}}
\newcommand{\Div}{\operatorname{Div}}
\newcommand{\ssag}[1]{\operatorname{\mathsf{SSAG}}_{q}\left(#1\right)}
\newcommand{\GRS}{\operatorname{\mathsf{GRS}}}
\newcommand{\gen}{\mathfrak{g}}
\newcommand{\degab}[1]{\deg_{a,b}\left(#1\right)}
%%%%%%%%%%%%%%%%%%%%%%%%%%%%%%%%%%%%%%%%%%%%%%%%%%%%%%
\newcommand\myeq{\mathrel{\stackrel{\makebox[0pt]{\mbox{\normalfont\tiny def}}}{=}}}
\setbeamertemplate{frametitle}{%
	\nointerlineskip%
	\begin{beamercolorbox}[wd=\paperwidth,ht=2.0ex,dp=0.6ex]{frametitle}
		\hspace*{1ex}\insertframetitle%
	\end{beamercolorbox}%
}
\makeatletter
\usepackage[symbol]{footmisc}
\renewcommand{\thefootnote}{\fnsymbol{footnote}}

%%%%%%%%%%%%%%%%%%%%%%%%%%%% Title %%%%%%%%%%%%%%%%%%%%%
\title[Goppa-Like AG codes Distinguisher]{Goppa–Like AG Codes From $C_{a,b}$ Curves And Their Behaviour Under Squaring Their Duals}
%\subtitle{Subtitle}
\author[SEK]{\textbf{Sabira El Khalfaoui} \vspace{0.3cm}\\\vspace{1cm} \textit{\textcolor{ballblue}{Joint work with}} \textbf{Mathieu Lhotel, Jade Nardi}}



\institute[]{\large{\textbf{IRMAR} \\ \smallskip \textit{\textcolor{ballblue}{GAE seminar}}}}
\date[Mai 2023]

\begin{document}
	
	\maketitle
	
	\begin{frame}{Table of contents}
		\setbeamertemplate{section in toc}[sections numbered]
		\tableofcontents%[hideallsubsections]
	\end{frame}
	
	\section{Preliminaries}
	
	\begin{frame}
		\frametitle{Linear code}
		Let $q$ be a prime power and $m \geq 1$. 
		\begin{block}{Definition: Linear code}
			A \textbf{linear code} $\calC$ is a vector subspace of $\fq^n$. $[n,k,d]_{q}$ (or $[n,k]_{q}$) denotes its parameters.
		\end{block}

	 Its \textbf{dual code} $\calC^{\perp}$ is a $[n,n-k]_{q}$ code defined by 
	\[\calC^{\perp}=\left\lbrace \mathbf{x} \in \fq^n \mid \mathbf{c} \cdot \mathbf{x}=0, \text{ for all } \mathbf{c} \in \calC \right\rbrace, (\text{$\cdot$ is the usual scalar product.})\]  
	
		\begin{block}{Definition: \textbf{Generalized Reed--Solomon (GRS)}}
			Let $\mathbf{x}=(x_1,\cdots,x_n) \in \fq^n$ be a vector with distinct entries (\textbf{support}) and \textbf{multiplier} $\mathbf{y} \in (\fq^*)^n$.
			\vspace{-0.7em}
			\[\GRS_r(\mathbf{x},\mathbf{y})=\{(y_1f(x_1),y_2f(x_2),\dots,y_nf(x_n)) \mid f \in \fq[X] \text{ such that } \deg f < r \}.\]
		\end{block}
	\begin{tcolorbox}[colback=aliceblue]
		Let $\mathbf{x}=(x_1,\cdots,x_n) \in \fq^n$ be a (\textbf{support}) and $\mathbf{y} = \mathbf{1}$.
		\vspace{-0.7em}
		\[RS_r(\mathbf{x})=\{(f(x_1),f(x_2),\dots,f(x_n)) \mid f \in \fq[X] \text{ such that } \deg f < r \}.\]
	\end{tcolorbox}
	\end{frame}

\begin{frame}
	\frametitle{Subfield Subcode}
	\begin{block}{Definition: Subfield subcode}
		Let $\calC\subseteq \fqm^n$ be a linear code.
		Its \textbf{subfield subcode} $\calC|_{\fq}$ is defined by 
		\[\calC|_{\fq}=\calC \cap \mathbb{F}_q^n.\]
	\end{block}
\begin{block}{Definition: Goppa code}
	Let $\mathbf{x}=(x_1,\cdots,x_n) \in \fqm^n$ be a vector with distinct entries and $g \in \fqm [x]_{<r}$ be a polynomial such that $\forall i, g(x_i)\neq 0$, the \textbf{Goppa code} associated to $(\mathbf{x}, g)$ is defined as: \[\Gamma_r(\mathbf{x},g)= \GRS_r(\mathbf{x},g(\mathbf{x})^{-1})^\perp|_{\fq}.\]
	where $g(\mathbf{x})^{-1}=(g(\mathbf{x_1})^{-1},\dots,g(\mathbf{x_n})^{-1})$
\end{block}
\end{frame}


\begin{frame}
	\frametitle{Algebraic geometry (AG) code}
	Let $\calX$ be a \textbf{smooth and irreducible projective curve} over $\fq$ of \textbf{genus} $\mathfrak{g}$.
	
	\begin{itemize}
		\item Let $\calP = \left\lbrace P_1,\cdots,P_n \right\rbrace$ be an $n$-tuple of pairwise distinct $\fq$\textbf{-rational points} of $\calX$.
		\item A \textbf{divisor} $G$ on $\calX$ is $G=\sum n_P P$, for almost all $n_P=0$.
		\item Its \textbf{degree} is $\deg G = \sum n_P$ and \textbf{support} $\Supp(G)=\left\lbrace P \in \calX \, |\, n_P\neq 0\right\rbrace $.
		\item Let $G$ be an $\fq$\textbf{-divisor} of $\calX$ such that $\Supp(G) \cap \calP= \varnothing$.
		\item The \textbf{Riemann--Roch space} of a \textbf{divisor} $G$ is a $\fq$ vector space
		$$ \calL(G) := \set{f \in \fq(\calX) \mid (f) + G \geq 0} \cup \set{0}.$$
	\end{itemize}
\begin{tcolorbox}[colback=lightkhaki]
	We have an \textbf{evalution map} $ev_{\calP} \colon \calL(G)	\rightarrow   \fq^n $ defined by $ev_{\calP}(f)=\left( f(P_1),\dots,f(P_n) \right)$.
%	\vspace{-1.2em}
%	\[ev_{\calP} \colon \left\{
%	\begin{array}{ccc}
%		\calL(G)& 	\rightarrow&   \fqm^n\\      
%		f   &  \mapsto &   \left( f(P_1),\dots,f(P_n) \right) \end{array}\right. \]
\end{tcolorbox}
\begin{block}{Definition: AG code}
	The \textbf{AG code} associated to the triplet $(\calX, \calP, G)$ is:
	\vspace{-0.8em}
	\[ \calC_{\calL}(\calX,\calP,G)= \left\lbrace ev_{\calP}(f)=(f(P_1),\cdots,f(P_n))\,|\, f \in \calL(G)\right\rbrace. \] 
\end{block}
\vspace{-0.8em}
It is an $[n, k \geq \ell(G)]_{\fq}$ code with $n=\#\calP$, $\ell(G) = \dim\calL(G)$.
\end{frame}
\begin{frame}
	\frametitle{AG codes and their subfield subcodes }
%		\vspace{-0.8em}
%	\begin{tcolorbox}[colback= lightkhaki]
%		RS code is an AG code:
%		\vspace{-0.8em}
%		\[RS_r(\mathbf{x})= \calC_{\calL}(\calX,\calP,(r-1)P_{\infty}),\]
%		$\calX = \mathbb{P}^1$ over $\fq$, $\calP$ is the set of affine points of $\mathbb{P}^1$, $\calL((r-1)P_{\infty})=\Span{1,x,\cdots,x^{r-1}}$.
%		\vspace{-0.8em}
%	\end{tcolorbox}
\begin{block}{From Riemann-Roch Theorem \cite{Sti09}}
	We have 
	\[\ell(G)\geq \deg G +1-\mathfrak{g},\]
	with equality hold if $\deg (G) \geq 2\mathfrak{g} -1$.
\end{block}
		\begin{block}{Dual of AG code }
		The \textbf{dual code} of $C_{\calL}(\calX,\calP,G)$ is an AG code that satisfies:
		\[ C_{\calL}(\calX,\calP,G)^{\perp} = C_{\calL}(\calX,\calP,G^{\perp}),\]
		with $G^{\perp} = D_{\calP}-G+W$, where $D_{\calP} = \sum\limits_{P \in \calP} P$, and $W=(\omega)$ is a canonical divisor of degree $2\mathfrak{g} - 2$.
	\end{block}
	\vspace{-0.8em}
\begin{tcolorbox}[colback= aliceblue]
	The subfield subcode (\textbf{SSAG}) of an AG code $C_{\calL}(\calX,\calP,G)$ defined over $\fqm$ is
	\[C_{\calL}(\calX,\calP,G)\cap \fq^n.\]
\end{tcolorbox}
\end{frame}
%-----------------------------------------
\section{Schur product - Square code - Distinguisher}
%-----------------------------------------
\begin{frame}
	\frametitle{Schur Product and Square code Distinguisher}
	In $\fq^n$ we denote by $\star$ the \textbf{Schur product}:
	\[ \mathbf{c} \star \mathbf{d} \myeq (a_1b_1,\cdots,a_nb_n). \]
	%	\begin{block}{Definition: Schur product}
		%		The Schur product of two vectors $\mathbf{a}$,$\mathbf{b} \in \fqm^n$ is defined as 
		%		\[ \mathbf{a} \star \mathbf{b} := (a_1b_1,\cdots,a_nb_n). \]
		%	\end{block}
	Then, the \textbf{Schur product} of two code $\calC, \calD \subseteq \fq^n$:
	\[ \mathbf{\calC } \star \mathbf{\calD }\myeq \Span{\mathbf{c} \star \mathbf{d} \mid \mathbf{c} \in \calC, \mathbf{d} \in \calD}. \]
	%	\begin{block}{Schur Product of codes}
		%		If $\calC$ and $\calD$ are two codes over $\fqm$ with same length $n$, their Schur product is defined as the following component--wise product:
		%		\[ \calC \star \calD := \Span{\mathbf{c} \star \mathbf{d} \mid \mathbf{c} \in \calC, \mathbf{d} \in \calD}. \]
		%	\end{block}
	If $\calC = \calD$, then we denote by $\mathbf{\calC^2 \myeq \calC \star \calC} $.
	\begin{alertblock}{Distinguisher-Square code}
		Let $\calC$ be an $[n, k_{\calC}]_q$ random linear code, we expect that
		\[\displaystyle \dim_{\mathbb{F}_{q}}(\calC^{\star2}) \leq \mathrm{min}\left(n,\binom{k_{\calC}+1}{2}\right).\]
	\end{alertblock}
	We can use this property to identify certain linear codes from random ones.
	\begin{tcolorbox}[colback=aliceblue]
		 If $r \leq \frac{n+1}{2}$:
		$\dim_{\fq}(\GRS_r(\mathbf{x},\mathbf{y}))^{\star 2}=\dim_{\fq}(\GRS_{2r-1}(\mathbf{x},\mathbf{y\star y}))= 2r+1.$
	\end{tcolorbox}
\end{frame}
\begin{frame}
	\frametitle{Schur product of Algebraic Geometry codes}
	\begin{itemize}
		\item[\textrightarrow] The product of $\calL(F)$ and $\calL(G)$ is defined as:
		$\calL(F) \cdot \calL(G) := \Span{ f \cdot g : (f,g) \in \calL(F) \times \calL(G)}$.
		\item[\textrightarrow] For a given $f,g \in \calL(G)$, we have $ev_{\calP}(f.g)=ev_{\calP}(f)\star ev_{\calP}(g)$.
	\end{itemize}
	\begin{block}{{\cite[Theorem]{Mum70}}}
		Let $F,G$ be two divisors on $\calX$. Then
		\vspace{-0.9em}
		\[ \calL(F) \cdot \calL(G) \subseteq \calL(F+G),\]
		equality holds if $\deg(G) \geq 2\mathfrak{g}+1$ and  $\deg(F) \geq 2\mathfrak{g}$.
	\end{block}
\begin{alertblock}{Corollary: Schur product of AG codes}
	Let $F,G$ be two divisors on $\calX$ both with disjoint support with $\calP$. Then
	\vspace{-0.9em}
	\[C_{\calL}(\calX,\mathcal{P},F)\star C_{\calL}(\calX,\mathcal{P},G)\subseteq C_{\calL}(\calX,\mathcal{P},F+G).\]
	Equality holds if $\deg(G) \geq 2\mathfrak{g}+1$ and  $\deg(F) \geq 2\mathfrak{g}$.
\end{alertblock}
	\begin{itemize}
		\item For $G=F$ and $\deg(G) \geq 2\mathfrak{g}+1$, we have $C_{\calL}(\calX,\mathcal{P},G)^{\star2} = C_{\calL}(\calX,\calP,2G)$.
		\item For $\deg(G) \geq 2\mathfrak{g}+1$, we have $\dim_{\fq}(C_{\calL}(\calX,\mathcal{P},G)^{\star2})=2\deg G - \mathfrak{g} + 1$.
	\end{itemize}
\end{frame}
\section{Motivation}
\begin{frame}
	\frametitle{McEliece Cryptosystem}
	\textbf{McEliece cryptosysem} was the first code-based public key encryption introduced in 1978.	%was introduced in 1978. 
	Its \textbf{security} is based on \textcolor{ballblue}{i)} the hardness of \textbf{decoding} random linear codes \textcolor{ballblue}{ii)} the \textbf{indistinguishabilty} of the chosen codes from random ones.
	
\textbf{McEliece's original proposal} \cite{mceliece1978public} is based on \textbf{binary Goppa code} \textcolor{red}{\textrightarrow} Large key size.

\vspace{1em}

%\textcolor{applegreen}{\faCheckCircle} It is still resistant to attack. \textcolor{red}{\faExclamationTriangle} Large key size.
\textcolor{ballblue}{\textbf{How To Reduce The Key Size?}}
\vskip-0.7em
\begin{columns}[T,onlytextwidth]
	\column{0.6\textwidth}
	\begin{itemize}
		\item[\textcolor{ballblue}{\faIcon{file-alt}}] \textbf{GRS} codes with $m = 1$ by \textbf{Niederreiter} 1986.
		\item[\textcolor{red}{\faFrownOpen}] Broken by
		\textbf{Sidelnikov and Shestakov}.
		\item[\textcolor{ballblue}{\faIcon{file-alt}}] \textbf{AG codes} and \textbf{\textcolor{applegreen}{SSAG}} by \textbf{Janwa, Moreno} 1996.
		\item[\textcolor{red}{\faFrownOpen}]\textbf{ AG codes} for genus $\leq 2$ is broken by \textbf{Faure Minder} 2008. For any genus by \textbf{Couvreur, Marquez–Corbella, Pellikaan} 2014 - 2017
	\end{itemize}
	\column{0.4\textwidth}
	\begin{enumerate}
		\item[\textcolor{ballblue}{\faIcon{file-alt}}] \textbf{Subcodes} of \textbf{GRS codes} by \textbf{Berger, Loidreau} 2001.
		\item[\textcolor{red}{\faFrownOpen}] Broken by \textbf{Wieschebrink} 2010.
		\item[\textcolor{applegreen}{\faIcon{file-alt}}] In \cite{EKN21crypto} we suggest "the $\mathbb{F}_{q^2}/\fq$ \textbf{subfield subcodes} of $1$\textbf{-point Hermitian codes}".
	\end{enumerate}
\end{columns}
\begin{tcolorbox}[colback=aliceblue]
	\textcolor{applegreen}{\faIcon{file-alt}} In this work we propose to use \textbf{Goppa-like AG codes}.
\end{tcolorbox}
\end{frame}
\begin{frame}
	\frametitle{Goppa-like AG code construction VS Classical Goppa code}
\begin{columns}[c]
	\hspace{-30pt}
	\begin{column}{0.7\textwidth}
		\textbf{Goppa-like AG code}
		\begin{itemize}
		\item Let $\calL(D)$ be the \textbf{Riemann-Roch space} such that $D$ is an \textbf{effective divisor} ($D \geq 0$) of positive degree $s$ on the curve $\calX$ over $\fqm$.
		\item Let $g \in \fqm(\calX)$ such that $g \in \calL(D')$ where $\deg D' = s'$ and $s'>s$. 
		%\textcolor{ballblue}{\textrightarrow $g$ plays the role of the \textbf{multiplier} in \textbf{GRS} code}.
		\item Given $\calP \in \calX(\fqm)$ such that $\calP \cap \Supp(g) = \varnothing$ and $\calP \cap \Supp(D) = \varnothing$.
		%The AG code associated with the divisor $D+(g)$
		\item $\calL(D+(g)) = g^{-1}\calL(D)$. \textcolor{ballblue}{\textrightarrow $ev_{\calP}(fg^{-1})=\left( f(P_1)g^{-1}(P_1),\dots,f(P_n)g^{-1}(P_n) \right)$}. %\textcolor{ballblue}{\textrightarrow the \textbf{evalution map} $ev_{\calP} \colon \calL(D+(g))	\rightarrow   \fqm^n $ defined by $ev_{\calP}(fg^{-1})=\left( f(P_1)g^{-1}(P_1),\dots,f(P_n)g^{-1}(P_n) \right)$}.
		\item $\calC := \calC_{\calL}(\calX,\calP,D+(g))=\set{\left(f(P)g(P)^{-1}\right)_{P \in \calP} \mid f \in \calL(D)}.$
	\end{itemize}
		
		%%%%%%%%%%%%%%%%%%%%
		\begin{block}{The \textbf{Goppa--like} AG code associated to $\calC$}
		$$ \Gamma(\calP,D,g) := \calC^{\perp}|_{\fq}.$$
		\end{block}
	\end{column}
	\hspace{-30pt}
	\vrule{}
	\begin{column}{0.3\textwidth}
		
		\textcolor{ballblue}{\textbf{Classical Goppa codes}}
		\vspace{1em}
		\begin{itemize}
			\item $\fqm[x]_{<r}$,
			\vspace{1em}
			\item $g \in \fqm[x]_{= r}$,
			\vspace{0.5em}
			\item $\mathbf{x}=(x_1,\cdots,x_n)$,
			\vspace{0.5em}
			\item $\forall i, g(x_i)\neq 0$,
			\vspace{0.5em}
			\item $\GRS_r(\mathbf{x}, g(\mathbf{x})^{-1})$,
		\end{itemize}
	 \textcolor{ballblue}{$(g(x_1)^{-1}f(x_1),\dots,g(x_n)^{-1}f(x_n)) $}
	 \begin{itemize}
	 	\item $\Gamma_r(\mathbf{x},g)= \GRS_r(\mathbf{x},g(\mathbf{x})^{-1})^\perp|_{\fq}$
	 \end{itemize}
 \vspace{3em}
	\end{column}
\end{columns}
\end{frame}

\begin{frame}
	\frametitle{Motivation I: McEliece Cryptosystem - New Variants}
	\textbf{\textcolor{applegreen}{Variants based on Subfield Subcodes of AG codes}}
		\begin{table}[h]
		\begin{center}
			\scalebox{0.8}{
			\begin{tabular}{|c|c|c|c|c|c|c|}
				\hline
				$q$&  $s$ & $n$ & $k$ & $t$ & Security bits\footnote{\textbf{With respect to Prange algorithm}}& Key--Size(bit)\\
				\hline \hline
				
				${11}$	&1\,174 & 1\,331 & 927 & 78 & $142.33$ & $1\,123\,524$ \\
				
				\hline \hline
				${13}$&2\,039 & 2\,197 & 1\,735 & 79 & $185.89$ & $3\,206\,280$ \\
				
				\hline 
				${16}$&3\,980 & 4\,096 & 3\,634 & 58 & 187.40& 6\,715\,632   \\
				
				\hline \hline
				${13}$& 1\,861 & 2\,197 & 1\,398 & 168 & 263.01& 4\,468\,008\\
				
				\hline 
				${16}$& 3\,874 & 4\,096 & 3\,422 & 111 & 300.65 & 9\,225\,712\\
				\hline
			\end{tabular}}
			%\vspace*{0.1em}
			\caption{McEliece cryptosystem based on \textbf{1-point Hermitian subfield subcodes}}
		\end{center}
	\end{table}
\vspace*{-0.8em}
	\begin{table}[h]
		\begin{center}
			\scalebox{0.8}{
			\begin{tabular}{|c|c|c|c|c|c|c|}
				\hline
				$q$&  $s$ & $n$ & $k$ & $t$ & Security bits& Key--Size(bit)\\
				\hline \hline
				
				${11}$	&265&1320& 898& 77& \cellcolor{applegreen}\textbf{153}& 1\,136\,868 \\
				
				\hline \hline
				\rowcolor{applegreen} \textbf{13}&\textbf{312}&\textbf{2188}& \textbf{1718}& \textbf{77}& \textbf{198}& \textbf{2\,422\,380}  \\
				
				\hline 
				${16}$&354& 4078& 3608& 56& \cellcolor{applegreen}\textbf{199}& 6\,783\,040   \\
				
				\hline \hline
				\rowcolor{applegreen}\textbf{13}& \textbf{490}& \textbf{2189}& \textbf{1363}& \textbf{166}& \textbf{270}& \textbf{3\,377\,514} \\
				
				\hline 
				${16}$& 460 &4080& 3398& 109& \cellcolor{applegreen}\textbf{313}& 9\,269\,744\\
				\hline
			\end{tabular}}
			%\vspace*{0.1em}
			\caption{\textbf{Goppa-Like Hermitian codes }parameters $\Gamma(\calP,sP_\infty,g)$ over $\F_{q^2}$.} \label{table:goppa-herm}
		\end{center}
	\end{table}
\vspace*{-0.8em}
\end{frame}

\begin{frame}
	\frametitle{Motivation II: Mora and Tilich result \cite{MT21}}
	\begin{block}{Distinguishing Goppa codes for $r \geq q-1$ \cite{MT21}}
		
		\[\dim_{\fq}(\Gamma_r(\mathbf{x},g)^{\perp})^{\star 2}\leq \mathcolor{ballblue}{\underbrace{\binom{rm+1}{2}}}_{\text{\textcolor{red}{random code}}}  - \frac{m}{2}r((2e_{\Gamma}+1)r - 2(q-1)q^{e_{\Gamma}-1}-1),\]
		%\vspace{-0.8em}
		with $e_{\Gamma}= \lceil\log_q(\frac{r}{(q - 1)^2})+1\rceil$.
	\end{block}
%\cite{bernstein2019classic}
\begin{table}[h]
	\begin{center}
		\scalebox{0.7}{
			\begin{tabular}{|c|c|c|c|c|c|c|}
				\hline
				\textbf{McEliece parameter set}& \textbf{n}&\textbf{ m}& \textbf{r\footnote[1]{The code order suggested for Classical McEliece cryptosystem in \cite{bernstein2019classic}.} }&\textbf{ R\footnote[2]{The code rate suggested for Classical McEliece cryptosystem in \cite{bernstein2019classic}.}}& \textbf{Largest distinguishable r}&\textbf{ Corresponding R}\\
				\hline \hline
				
				mceliece348864& 3488& 12& 64& 0.77982& 12& 0.95872\\
				\hline
				mceliece460896& 4608& 13& 96 &0.72917& 12& 0.96615\\
				\hline
				mceliece6688128& 6688& 13& 128& 0.75120& 15& 0.97084\\
				\hline
				mceliece6960119& 6960& 13& 119& 0.77773& 16& 0.97011\\
				\hline
				mceliece8192128& 8192& 13& 128& 0.79688& 19& 0.96985\\
				\hline
		\end{tabular}}
		%\vspace*{0.1em}
		\captionsetup{margin=10pt,font=footnotesize}
		\caption{Comparison between \textbf{Classic McEliece} and \textbf{smallest distinguishable
			code rates} \cite{MT21}.}
	\end{center}
\end{table}
\vspace{-1.2em}
\textcolor{red}{\faFrownOpen} \textcolor{gamboge}{This threats the \textbf{CFS signature scheme} that is based on \textbf{very high rate Goppa codes.}}
\vspace{1.5em}
\end{frame}
%---------------------------------------
\section{Trace Code and Subfield Subcode}
%----------------------------------------
\begin{frame}
	\frametitle{Trace Code}
	\begin{block}{Trace operator on $\mathbb{F}_{q^m}$}
		\[\Tr{x} = x + x^q + ... + x^{q^{m-1}} \in \fq.\]
	\end{block}

Let $\calC \subseteq \fqm^n$. The \textbf{trace code} $\Tr{\calC}=\{(\Tr{c_1},\cdots,\Tr{c_n})\,|\, c\in \calC\}$ over $\fq$ is a \textbf{linear code} of length $n$ over $\fq$ with
%\begin{equation}
\[\dim_{\mathbb{F}_q} \Tr{\calC} \leq \min\{mk,n\}.\]	
%\end{equation}
\vspace{-1.2em}
\begin{block}{Delsarte's theorem}
	Let $\calC$ be a \textbf{linear code} of lenght $n$ over $\fqm$. Then
	\[\left(\calC|_{\fq}\right)^{\perp} = \Tr{\calC^{\perp}}.\]
\end{block}
Equivalently, we have $\Tr{\calC}= \left(\calC^{\perp}|_{\fq}\right)^{\perp}$.
\vspace{-0.5em}
\begin{tcolorbox}[colback=gold]
	\[\Tr{\GRS_r(\mathbf{x},g(x)^{-1})} =( \GRS_r(\mathbf{x},g(\mathbf{x})^{-1})^\perp|_{\fq})^\perp= (\Gamma_r(\mathbf{x},g))^{\perp}.\]
\end{tcolorbox}
\end{frame}
\begin{frame}
	\frametitle{Goppa codes distinguisher \cite{MT21}}
	\begin{block}{Proposition \cite{MT21}}
		Let $\calC$ be a linear code over $\fqm$. We have
		\vspace{-0.9em}
		$$Tr(\calC)^{*2}\subseteq \sum\limits_{i=0}^{\lfloor{m/2} \rfloor} \Tr{\calC\star \calC^{q^i}}$$
		where $\calC^{q^i} \myeq \{c^{q^i} \,| \, c \in \calC \}  $.
	\end{block}
%\vspace{-0.8em}
	We set $\calC = \GRS_r(\mathbf{x},\mathbf{y})$ and $\mathbf{y}=\frac{1}{g(\mathbf{x})}$. For $0\leq i\leq e$ where $e=\lfloor \log_q(r)\rfloor$.
	\begin{itemize}
		\item $\calC\star \calC^{q^i}=\GRS_{(r-1)(q^i+1)+1}(\mathbf{x},\mathbf{y^{q^i+1}})$.
		\item Let $f \in \fq[x]_{<(r-1)(q^i+1)+1}$, computing the Euclidean division of $f$ by $g^{q^i -q^{i-1}+1}$
		%\vspace{-0.7em}
		\[ \Tr{\frac{f}{g^{q^i +1}}}=\Tr{\frac{f'}{g^{q^i +1}}},\]
		%\vspace{-0.9em}
		where $\deg f' < r(q^i -q^{i-1}+1)<(r-1)(q^{i
			+1})+1$.
		
		$\Rightarrow \: \Tr{\GRS_{(r-1)(q^i+1)+1}(\mathbf{x},\mathbf{y^{q^i+1}})} = \Tr{\GRS_{r(q^i -q^{i-1}+1)}(\mathbf{x},\mathbf{y^{q^i+1}})} = T_i$.
		\item $T_i \subset T_{i+1}$...
	\end{itemize}
\end{frame}
\begin{frame}
	\frametitle{Goppa codes distinguisher \cite{MT21}}
	\begin{itemize}
		%\item $\sum\limits_{i=0}^{e} \Tr{\GRS_r(\mathbf{x},\mathbf{y})\star \GRS_r(\mathbf{x},\mathbf{y})^{q^i}} \subseteq \Tr{\GRS_{r(q^e -q^{e-1}+1)}(\mathbf{x},\mathbf{y^{q^e+1}})}$.
		\item $\dim\left( Tr(\calC)^{*2}\right) \leq \sum\limits_{i=0}^{\lfloor{m/2} \rfloor} \dim \left( \Tr{\calC\star \calC^{q^i}}\right).$
		%\item $\dim_{\fq}\Tr{\calC\star \calC^{q^i}}\leq m\dim_{\fqm} \calC^2$ for $i \in \{e+1,\cdots, \lfloor{m/2} \rfloor\}$.
	\end{itemize}
	\vspace{-0.9em}
\begin{align*}
	\dim_{\fq}Tr(\calC)^{*2}\leq& \dim_{\fq}\Tr{\GRS_{r(q^e -q^{e-1}+1)}(\mathbf{x},\mathbf{y^{q^e+1}})}+ \sum\limits_{i=e+1}^{\lfloor{m/2} \rfloor} \dim_{\fq}\Tr{\calC\star \calC^{q^i}}\\
	\leq& mr(q^e -q^{e-1}+1) + \left( \frac{m-2}{2}-e\right) mr^2\\
	\leq& \mathcolor{ballblue}{\underbrace{\binom{rm+1}{2}}}_{\text{\textcolor{red}{random code}}}  - \frac{m}{2}r((2e+1)r - 2(q-1)q^{e-1}-1).
\end{align*}
\vspace{-0.5em}
\begin{itemize}
	\item $\dim_{\fq}Tr(\calC)^{*2} = \dim_{\fq}(\Gamma_r(\mathbf{x},g)^{\perp})^{\star 2}\leq \mathcolor{ballblue}{\underbrace{\binom{rm+1}{2}}}_{\text{\textcolor{red}{random code}}}  - \frac{m}{2}r((2e_{\Gamma}+1)r - 2(q-1)q^{e_{\Gamma}-1}-1)$,\vspace{-0.7em}
	with $e_{\Gamma}= \lceil\log_q(\frac{r}{(q - 1)^2})+1\rceil$ \cite{MT21}.
\end{itemize}
%	\begin{tcolorbox}[colback=aliceblue]
%		For $r\geq q-1$ \cite{MT21},
%		\vspace{-0.9em}
%		\[\dim_{\fq}(\Gamma_r(\mathbf{x},g)^{\perp})^{\star 2}\leq \mathcolor{ballblue}{\underbrace{\binom{rm+1}{2}}}_{\text{\textcolor{red}{random code}}}  - \frac{m}{2}r((2e_{\Gamma}+1)r - 2(q-1)q^{e_{\Gamma}-1}-1),\]
%		%\vspace{-0.8em}
%		with $e_{\Gamma}= \lceil\log_q(\frac{r}{(q - 1)^2})+1\rceil$.
%		\vspace{-0.6em}
%	\end{tcolorbox}
\end{frame}
%--------------------------------------------------------------
\section{1--point Goppa--Like AG Code On $C_{a,b}$ curve}
%--------------------------------------------------------------
\begin{frame}
	\frametitle{$C_{a,b}$ curve}
	\begin{block}{Definition \cite{Miu93}}
		Let $a,b$ be coprime positive integers. A $C_{a,b}$ curve over $\fqm$ is a curve $\calX_{a,b}$ having an irreducible, affine and non--singular plane model with equation
		%\begin{equation} \label{eq:equation_C_ab}
		\[f_{a,b}(x,y) = \alpha_{0a}y^a + \alpha_{b0}x^b + \sum_{ai+bj < ab} \alpha_{ij}x^iy^j = 0,\]	
		%\end{equation}
		with $\alpha_{0a}$ and $\alpha_{b0} \neq 0$.
	\end{block}
	
	\begin{itemize}
		\item It has a unique point at infinity, denoted by $P_{\infty}$. Its genus is $\mathfrak{g}_{a,b}:=\dfrac{(a-1)(b-1)}{2}.$
		%Riemann--Roch space $\calL(s P_\infty)$ has an explicit basis as follows:
		\item $\calL(sP_{\infty}) = \Span{x^iy^j \mid 0 \leq i, 0\leq j\leq a-1 \ \mathrm{and} \ ai+bj \leq s}.$
		 %Given a monomial of the form $x^iy^j$, we define its weighted degree by
		\item $\degab{x^iy^j} := ai+bj.$
	\end{itemize}
\end{frame}
\begin{frame}
\frametitle{One-point Goppa-like AG code on $C_{a,b}$-curves}
\begin{block}{Definition: One--point Goppa--like AG code on $C_{a,b}$-curves}
	Let $s'>s$ be two integers such that there exists a function $g \in \calL(s'P_\infty)$ with $\degab{g}=s'$. Given a set of points  $\calP \subset \calX_{a,b}(\F_{q^m})$ such that $\calP \cap \Supp(g) = \varnothing$, we define the \textbf{one--point Goppa--like AG code} associated to $\calP,s$ and $g$ as 
	\[\Gamma(\calP,sP_\infty,g) := \calC_{\calL}(\calX_{a,b},\calP,(sP_\infty+(g))^{\perp})|_{\fq}.\]
\end{block}
\begin{block}{Proposition \cite{GH00}}
	For any function $f \in \bigcup_{s \geq 0} \calL(s P_\infty)$, we can write $f=f_1g+f_2$ with 
	\[f_2 \in \calR(g):= \Span{x^u y^v \mid u \leq \beta + b-1 \text{ and } v\leq a-1 \text{ not both }  u \geq \beta \text{ and } v \geq \alpha}.\]
	Moreover, we have $\degab{f_2} \leq \degab{f}$ and $\dim_{\fqm} \calR(g) = \degab{g}.$ 
\end{block}

\end{frame}
\begin{frame}
	\frametitle{One--point Goppa--like AG code on $C_{a,b}$-curves}
	
	\begin{block}{Lemma \cite{lhotel2023goppa}}
		Take $i \geq 1$ and $s'>s \geq 0$. Let $g \in \calL\left(s'P_\infty\right)$ and $f \in \calL\left(\left(s'(q^i+1)-1\right)P_\infty\right)$. Then there exists $f' \in \calR\left(g^{q^i-q^{i-1}+1}\right)$ such that $\Tr{\frac{f}{g^{q^i+1}}} = \Tr{\frac{f'}{g^{q^i+1}}} $.
	\end{block}

	We set $\calC = \calC_{\calL}(\calX_{a,b},\calP,(sP_\infty+(g))$, $D = sP_\infty+(g)$.
	\begin{itemize}
		\item $\calL(D).\calL(D)^{q^i}\subseteq \calL((q^i +1)D) \implies \calC\star\calC^{q^i} \subseteq \calC_{\calL}(\calX_{a,b},\calP,(q^i +1)D)$.
		\item[\textrightarrow] $\mathcolor{ballblue}{\calC\star \calC^{q^i}=\GRS_{(r-1)(q^i+1)+1}(\mathbf{x},\mathbf{y^{q^i+1}}).}$
		\item $\Tr{\calC\star\calC^{q^i}} \subseteq \Tr{\calC_{\calL}(\calX_{a,b},\calP,(q^i +1)D)}$, this implies $$\dim_{\fq}(\Tr{\calC\star\calC^{q^i}}) \leq m(s(q^i +1)-\mathfrak{g} +1)$$.
		\vspace{-0.7em}
		\item $\Tr{\frac{f}{g^{q^i+1}}} = \Tr{\frac{f'}{g^{q^i+1}}}$ with $\degab{f}<\degab{f'}$.
		\item[\textrightarrow] $\mathcolor{ballblue}{f \in \fq[x]_{<(r-1)(q^i+1)+1}, \Tr{\frac{f}{g^{q^i +1}}}=\Tr{\frac{f'}{g^{q^i +1}}}, \deg f'<r(q^i -q^{i-1}+1).}$
	\end{itemize}
\end{frame}

\begin{frame}
	\frametitle{One--point Goppa--like AG code on $C_{a,b}$-curves}
	\begin{itemize}
		\item For $\mathcolor{gold}{s \geq (s'-s)q+2\mathfrak{g}_{a,b}-1}$, and $e\in \{0,\cdots,\lfloor \frac{m}{2}\rfloor\}$, we have
		\vspace{-0.6em}
		
	 $$ \sum \limits_{i=0}^{e}\Tr{\calC\star\calC^{q^i}} \subseteq \Tr{\calR(g^{q^{i}-q^{i-1}+1})}.$$
	 \item[\textrightarrow] $\mathcolor{ballblue}{\sum\limits_{i=0}^{e} \Tr{\GRS_r(\mathbf{x},\mathbf{y})\star \GRS_r(\mathbf{x},\mathbf{y})^{q^i}} \subseteq \Tr{\GRS_{r(q^i -q^{i-1}+1)}(\mathbf{x},\mathbf{y^{q^i+1}})}}$
		\item $\dim_{\fq}\sum \limits_{i=0}^{e}\Tr{\calC\star\calC^{q^i}} \leq ms'(q^e - q^{e-1} +1)$.
	\end{itemize}
\vspace{-0.7em}
\begin{align*}
	\dim_{\fq}Tr(\calC)^{*2} &\leq \dim_{\fq}\sum \limits_{i=0}^{e}\Tr{\calC\star\calC^{q^i}} + \dim_{\fq}\sum \limits_{i=e+1}^{\lfloor \frac{m}{2}\rfloor}\Tr{\calC\star\calC^{q^i}}\\
	&\leq \binom{mk+1}{2} - \dfrac{m}{2}(k^2(2e+1)+k-2s'(q^{e}-q^{e-1}+1)).
\end{align*}
\end{frame}


\begin{frame}
	\frametitle{Distinguisher of Goppa--like AG code on $C_{a,b}$-curves}
	\begin{block}{Theorem \cite{lhotel2023goppa} }
		Let $\mathcolor{gamboge}{s \geq (s'-s)q+2\mathfrak{g}_{a,b}-1}$ and $e^* := \min\left(\left\lfloor \frac{m}{2} \right\rfloor, \left\lceil \log_q\left(\frac{k^2}{s'(q-1)^2}\right)\right\rceil+1\right)$. Then
		\vspace{-0.7em}
		$$\dim_{\fq} (\Gamma(\calP,sP_\infty,g)^{\perp})^{\star 2}\leq \mathcolor{ballblue}{\underbrace{\binom{mk+1}{2}}}_{\text{\textcolor{red}{random code}}} - \dfrac{m}{2}(k^2(2e^*+1)+k-2s'(q^{e^*}-q^{e^*-1}+1)). $$
	\end{block}
\vspace{-0.7em}
\textcolor{ballblue}{\textbf{Sharpness of the bound}}
\vspace*{-0.7em}
	\begin{itemize}
		\item Let $\calX_{2,3}$ be an elliptic curve over $\fqm = \F_{3^6}$ defined by $ y^2+y = x^3+x+2.$
		\item $s'=s+1$ for $s \geq 0$, and $g \in \calL(s'P_{\infty})$. We set $\calC_g := \Gamma(\calP,sP_\infty,g)$.
	\end{itemize}
	\vspace*{-0.7em}
	\begin{table}[h]
		\begin{center}
			\scalebox{0.8}{
			\begin{tabular}{|c|c|c|c|c|c|}
				\hline
				n &$s$&$\dim_{\fq}\calC_g$ & $\dim_{\fq}((\calC_g)^{\star 2})$&$\dim_{\fq}((\calC_g^{\perp})^{\star2})$ & Upper bound\\
				\hline \hline
				$781$ &$4$& $757$& $781$&$234$ & $234$ \\
				\hline 
				$783$ &$5$& $753$& $783$ &$327$ & $327$   \\
				\hline \hline
				$782$ &$6$& $746$&$782$ &$402$ & $402$  \\
				\hline
				$783$ &$7$& $741$& $783$&$483$ & $483$  \\
				\hline \hline
				$782$ &$8$& $734$& $782$&$570$ & $570$   \\
				\hline
				$782$ &$9$& $728$& $782$&$663$ & $663$ \\
				\hline
				$781$ &$10$& $721$ & $781$&$762$ & $762$ \\
				\hline
			\end{tabular}
			}
			\caption{Sharpness of the bound.} \label{table:expl_sharpness}
		\end{center}
	\end{table}
\end{frame}
\begin{frame}
	\frametitle{High rate distinguishable codes: Elliptic curve case }
	
	\begin{itemize}
		\item Let $\Gamma(\calP,sP_\infty,g)$ be a one-point Goppa-like AG code on the elliptic curve  $\calX_{2,3}$ over $\fqm$.
		\item Compute the maximal distinguishable value of $s$ with $s'=s+1$ and $g\in \calL(s'P_{\infty})$.
	\end{itemize}
	
	\begin{table}[h]
		\begin{center}
			\scalebox{0.9}{
			\begin{tabular}{|c|c|c||c|c|c|c|c|c|}
				\hline
				$q$ & $m$ & $n$ & Largest distinguishable $s$ & Corresponding rate\\
				\hline \hline
				$2$ & $12$ & $4218$ & $14$ & $0,963$ \\
				\hline 
				$2$ & $13$ & $6688$ & $18$ & $0,982$  \\
				\hline \hline
				$3$ & $7$ & $2186$ & $15$ & $0,962$ \\
				\hline
				$3$ & $8$ & $6393$ & $24$ & $0,977$ \\
				\hline \hline
				$5$ & $5$ & $3043$ & $27$ & $0,961$  \\
				\hline
				$5$ & $6$ & $4500$ & $22$ & $0,971$ \\
				\hline
				$5$  & $6$ & $6688$ & $30$ & $0,976$ \\
				\hline \hline
				$7$ & $4$ & $2395$ & $27$ & $0,957$ \\
				\hline
				$7$ & $5$ & $4650$ & $26$ & $0,971$ \\
				\hline
				$7$ & $5$ & $8192$ & $37$ & $0,979$ \\
				\hline \hline
				$17$ & $3$ & $4820$ & $92$ & $0,943$ \\
				\hline
			\end{tabular}
		}
			\vspace*{0.3em}
			\caption{Largest distinguishable Goppa--like AG code in elliptic case.}
		\end{center}
	\end{table}
	
\end{frame}
\section{Analysis and Application}
%------------------------------------------------
\begin{frame}
	\frametitle{Analysis of the distinguisher}
	\textcolor{applegreen}{\textbf{How can we choose appropriate parameters for McEliece Cryptosystem}?}
	\begin{itemize}
		\item Let $\calC := C_\calL(\calX_{a,b},\calP,sP_\infty +(g))$, with $\degab{g} = s'>s\geq 2\mathfrak{g}_{a,b}-1$. If $\mathcolor{gold}{s \geq (s'-s)q+2\mathfrak{g}_{a,b}-1}$ , then
		%\begin{equation} \label{eq:best_upper_bound}
		\[ \dim_{\fq} (\Gamma(\calP,sP_\infty,g)^{\perp})^{\star 2} \leq \min \left(\frac{m}{2}\left(2s'(q^{e^*}-q^{e^*-1}+1)+k^2(m-1-2e^*)  \right),n\right),\]	
		%\end{equation}
		where $e^* := \left\lceil \log_q\left(\dfrac{k^2}{s'(q-1)^2}\right)\right\rceil+1$.
		\item \textbf{Hasse--Weil bound }
		\[\color{ballblue}{n = \# \calP = |\calX_{a,b}(\fqm)|-2 \leq q^m-1+2\mathfrak{g}_{a,b}\sqrt{q^m}},\]
		\item The \textbf{parameters} have to be chosen such that 
		%\begin{equation} \label{eq:cond_not_to_distinguish}
		\[\color{ballblue}{\frac{m}{2}\left(2s'(q^{e^*}-q^{e^*-1}+1)+k^2(m-1-2e^*)  \right)\geq q^m-1+2\mathfrak{g}_{a,b}\sqrt{q^m}}.\]	
		%\end{equation}
		
	\end{itemize}
\end{frame}
\begin{frame}
	\frametitle{One-point Goppa like AG codes on Hermitian curves}
	\begin{block}{Definition }
		Let $m \geq 1$ be an even integer and $q_0 := q^{m/2}$.The \textbf{Hermitian curve} $\calH$ over $\fqo$ is defined by the equation
		\[\calH : y^{q_0}+y = x^{q_0+1}.\]
		Its genus is given by $\mathfrak{g}_{\calH} = \frac{q_0(q_0-1)}{2}$ and it is a \textbf{maximal curve}, \emph{i.e.} $\#\calH(\fqo) = q_0^3+1$.
	\end{block}
	
	\begin{block}{Proposition \cite{lhotel2023goppa}} 
		Suppose $\mathcolor{gamboge}{s \geq (s'-s)q+2\mathfrak{g}_{a,b}-1}$. Then for any choice of $g$ and $\calP$, the \textbf{one--point Goppa--like Hermitian code} $\Gamma(\calP,sP_\infty,g)$ resists the given distinguisher.
	\end{block}
\begin{tcolorbox}[colback=aliceblue]
	$$\dim_{\fq} (\Gamma(\calP,sP_\infty,g)^{\perp})^{\star 2}\leq \mathcolor{ballblue}{\underbrace{\binom{mk+1}{2}}}_{\text{\textcolor{red}{random code}}} - \dfrac{m}{2}(k^2(2e^*+1)+k-2s'(q^{e^*}-q^{e^*-1}+1)). $$
\end{tcolorbox}
\end{frame}
	\begin{frame}[allowframebreaks]{References}
		
		\bibliography{demo}
		\bibliographystyle{alpha}
		
	\end{frame}
	
\end{document}
