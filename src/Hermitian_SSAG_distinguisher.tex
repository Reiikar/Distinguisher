\documentclass[a4paper]{article}

%%%%%%%% BIBLIOGRAPHIE %%%%%%%%
%\usepackage[backend=biber]{biblatex} %Imports biblatex package  //version biber et pas biblatex, ne marche pas direct sur texmaker.
%\addbibresource{biblio.bib} %Import the bibliography file

%---------Package---------------%

\usepackage[utf8]{inputenc}
\usepackage{xcolor}
\usepackage{amsmath}
\usepackage{amssymb}
\usepackage{amsfonts}
\usepackage{verbatim}
\usepackage{amsthm}
\usepackage{geometry}
\geometry{hmargin=2cm,vmargin=1.5cm}
\usepackage{hyperref} %Références cliquables

%%%%%---Pour les dessins---%%%%%

\usepackage{tikz}
\usetikzlibrary{shapes,positioning}
\usetikzlibrary{calc}
\usetikzlibrary{matrix,arrows,patterns.meta}
\usepackage{graphicx}

\tikzset{c/.style={every coordinate/.try}}

%--------Theorems-------------%

%\theoremstyle{plain}
\newtheorem{thm}{Theorem}[section]
\newtheorem{proposition}[thm]{Proposition}
\newtheorem{coro}[thm]{Corollary}
\newtheorem{lemma}[thm]{Lemma}

\theoremstyle{definition}
\newtheorem{definition}[thm]{Definition}
\newtheorem{example}{Example}

\theoremstyle{remark}
\newtheorem{remark}[thm]{Remark}


%------------Commands-----------%

\newcommand{\calA}{\mathcal{A}}
\newcommand{\calB}{\mathcal{B}}
\newcommand{\calS}{\mathcal{S}}
\newcommand{\calP}{\mathcal{P}}
\newcommand{\calH}{\mathcal{H}}
\newcommand{\calL}{\mathcal{L}}
\newcommand{\calC}{\mathcal{C}}
\newcommand{\calD}{\mathcal{D}}
\newcommand{\calO}{\mathcal{O}}
\newcommand{\calR}{\mathcal{R}}
\newcommand{\calT}{\mathcal{T}}
\newcommand{\calX}{\mathcal{X}}
\newcommand{\fqm}{\mathbb{F}_{q^m}}
\newcommand{\fq}{\mathbb{F}_{q}}
\newcommand{\fqo}{\mathbb{F}_{q_0^2}}
\newcommand{\F}{\mathbb{F}}
\newcommand{\Z}{\mathbb{Z}}
\newcommand{\PP}{\mathbb{P}}
\newcommand{\R}{\mathbb{R}}
\newcommand{\Tr}[1]{\operatorname{Tr}_{\mathbb{F}_{q^m}/\fq}\left(#1\right)}
\newcommand{\set}[1]{\left\{#1\right\}}
\newcommand{\Floor}[1]{\left\lfloor #1 \right\rfloor}
\newcommand{\Span}[1]{\operatorname{Span}\left(#1\right)}
\newcommand{\LT}[1]{\operatorname{LT}\left(#1\right)}
\newcommand{\Supp}{\operatorname{Supp}}
\newcommand{\Div}{\operatorname{Div}}
\newcommand{\ssag}[1]{\operatorname{\mathsf{SSAG}}_{q}\left(#1\right)}

\newcommand{\GRS}{\operatorname{\mathsf{GRS}}}

\newcommand{\gen}{\mathfrak{g}}
\newcommand{\degab}[1]{\deg_{a,b}\left(#1\right)}


%%----Commentaires----%%%

\newcommand\jade[1]{\textcolor{purple}{#1}}
\newcommand\TODO[1]{\textcolor{red}{TO DO: #1}}
\newcommand\Comment[1]{\textcolor{red}{Comment: #1}}
\newcommand\mathieu[1]{\textcolor{brown}{#1}}
\newcommand{\modifier}[1]{\textcolor{red}{\textbf{#1}}}
\newcommand\sabira[1]{\textcolor{blue}{#1}}
%%%%%-------- Keywords--------------
\providecommand{\keywords}[1]{\textbf{Keywords---} #1}

%-----Début---------%

\title{Goppa--like SSAG code from $C_{a,b}$ curves and their behaviour by squaring their dual}
\author{Mathieu Lhotel, Sabira El Khalfaoui, Jade Nardi}
\date{}

\begin{document}

%\parindent=0pt % Aucune indentation.

\maketitle

\TODO{SABIRA: classification MSC}

\begin{abstract}
In this paper, we introduce a family of codes than can be used in the McEliece cryptosystem based, called \emph{Goppa--like AG codes}. These are a generalization of Goppa codes constructed from any curve of genus $\mathfrak{g} \geq 0$. Focusing on codes from $C_{a,b}$ curves, we study the behaviour of the dimension of the square of their dual to determine their resistance to distinguisher attacks similar to the one for alternant and Goppa codes developed by Mora and Tillich \cite{MT21}. We propose numerical experiments to measure how sharp is our bound.
\end{abstract}
	
\keywords{AG codes, $C_{a,b}$ curves, Goppa--like AG codes, Trace codes, Schur product.}

\section*{Introduction}

McEliece cryptosystem is the one of the last code-based candidates for standardisation of post--quantum cryptrographic to the NIST competition since the third round. It guarantees the smallest ciphertexts among all the candidates, but it suffers from the largest public keys. Over the past forty years, there were many attempts in replacing the family of binary Goppa codes by other structured families of codes to reduce the key size.

The security of McEliece cryptosystem is based on two assumptions: $(i)$ in the average it is hard to decode $t$ errors for codes having the same parameters as the codes used as key and $(ii)$ codes used as key must be hardly distinguishable from arbitrary ones. When proposing another family of codes than binary Goppa codes, one must ensure that both of these assumptions are still valid.

Algebraic geometry (AG) codes appear to be good candidates for McEliece cryptosystem, since they are defined as evaluation codes on any curve (rather than only consedering the projective line in the case of GRS codes). Moreover, they also come with an efficient decoding algorithm (see the survey of Høhold and Pellikaan \cite{HP95}).
In 1996, Janwa and Moreno \cite{JM96} proposed to use AG codes, concatenation or subfield subcode of these codes in McEliece cryptosystem. As for concatened ones, they were broken by Sendrier \cite{Sen94}. For AG codes, Faure and Minder proposed an attack in \cite{FM08,Min07,Fau09} for codes from genus $\geq 2$ curves. The scheme based on AG codes have been completely broken by Couvreur, Marquez-Corbella and Pellikaan \cite{CMR17} who proposed a filtration--based attack on AG codes for any genus, enabling decoding just by handling the public key and without knowledge of the curve and/or the divisor. However, the authors underlined that subfield subcodes of AG codes (SSAG codes in short) are resistant to this filtration attack. Moreover, some of these codes have a good designed minimum distance: like Cartier codes, see \cite{Cou14}. This resistance to structural attacks, as well as their good parameters motivated several works on SSAG codes. Recently, Zhao and Chen \cite{ZC22} analysed the parameters and the decoding performance of one-point elliptic subfield subcodes, showing that in the binary case, decoding results on these codes outperform the similar rate BCH codes. Also, the authors in \cite{PJ14,EKN21} focus on one--point Hermitian codes, and manage to compute the exact dimension of their subfield subcode. 

\medskip

Let $\fqm$ be a finite extension of the finite field $\fq$. Given an AG code $\calC :=C_{\calL}(\calX,\calP,G) \subseteq \fqm^n$, its subfield subcode over $\fq$ is defined by 
$$\ssag{\calX,\calP,G} := \calC \cap \fq^n.$$  

In this paper, we introduce a specific class of SSAG codes, namely Goppa--like AG codes. The idea of this construction is to introduce  some randomness into the family of SSAG codes drawn as private key by mimicking the role of the Goppa polynomial in the case of classical Goppa codes. Given an effective divisor $D\in \Div(\calX)$ on the curve, we will consider SSAG codes associated with divisors of the form $D+(g)$, for a large family of functions $g \in \fqm(\calX)$ on the curve. A Goppa--like AG code is then defined as the subfield subcode over $\fq$ of $C_{\calL}(\calX,P,D+(g))^{\perp}$. This construction coincides with classical Goppa codes in the genus 0 case.%, since the latter codes are defined as the subfield subcode of the dual of some GRS code.
With a good choice of the curve and divisor, it possible to encode and decode these codes in a timely manner, this generalization has the potential to significantly improve on the original McEliece proposal. This encourages the present study of structural attacks on Goppa--like AG codes. 

Our starting point is the paper by Mora and Tilich \cite{MT21}, which benefited from the trace structure of the dual code of a subfield subcode to display a distinguisher for high rate Goppa codes. Their techniques relied on two features of GRS codes. First, their behaviour with respect to the Schur product is well-known. From this, the authors gave a sharp upper bound for the dimension of the square of the dual codes of alternant codes. In the particular case of Goppa codes, where the multiplicator has the form $\mathbf{y}=(g(x_1)^{-1},g(x_2)^{-1},\dots,g(x_n)^{-1})$ for some polynomial $g$ of degree $r$, they managed to get an even sharper upper bound by performing euclidean division by powers of $g$.

As Goppa--like AG codes extend Goppa codes, it is natural to wonder if one can derive a structural attack on these codes from Mora and Tillich's attack. The genus of the curve plays a significant role in the parameters of AG codes. The greater the genus, the more $\F_q$--points the curve $\calX$ may have and so the longer the code can be. But the parameters $[n,k,d]$ of an AG code satisfies $n-g+1 \leq k+d \leq n+1$, which means that an AG code is $g$--far form optimality. Also, the correction capacity naturally suffers from a big genus: the naive correction algorithm can correct up to $\frac{d-1-g}{2}$ errors. Only refined techniques, based on error locating pairs \cite{CP20}, manage to remove the term related to the genus. Therefore, caution should be exercised when it comes to the impact of the genus on the properties of AG codes.
Even if Mora and Tillich's attack does not threaten the Goppa codes used in the NIST competition, SSAG-based propositions may be vulnerable to a similar structural attack. 

In the present work, we are interested in the case of Goppa--like one--point AG codes from $\calC_{a,b}$ curves, a class of curves introduced in \cite{Miu93}. Since then, these curves have been extensively studied. They are especially interesting as we know an explicit monomial basis of the AG code associated with any multiple of its unique point at infinity, allowing us to efficiently encode one-point codes  efficiently \cite{BRS21}. Furthermore, they remains quite general: for examples, elliptic curves, Kummer or Artin-Schreier curves are $C_{a,b}$ curves. It is also natural to wonder how the genus affect the distinguisher: in particular, we give a sufficient condition of the minimal degree our divisor has to satisfy in order to mount the distinguisher. This bound is increasing with the genus of the curve, and coincides with the one given in \cite{MT21} in the case of classical Goppa codes. Consequently, when the genus gets higher, we are not able to distinguish codes associated to low degree divisor. In the worst cases, we might no be able to distinguish anything.

AG codes, as generalization of GRS codes, have exactly the same behaviour with respect to the Schur product. Moreover, some well-chosen AG codes are defined by the evaluation of \textit{multivariate} polynomials. In this case, we prove that performing division algorithm via Gröbner bases enable us to estimate the dimension of the square of the dual of Goppa--like AG codes. Even better, computations tend to show that the bound we obtain on the dimension is sharp whenever the Goppa--like code seems random (\emph{i.e} the function $g$ is sufficiently random). The counterpart is, as it was the case in \cite{MT21} for classical Goppa codes, that we can only distinguish high rate codes (more precisely, our maxima distinguishable rate are roughly the same). A comparison of their and our result is carried out in the case on Goppa--like codes on an elliptic curve. However, as previously discussed, if the genus becomes too large, the distinguisher may become ineffective: in particular, we show that Goppa--like AG codes constructed from the Hermitian curve resist our distinguisher, which is encouraging if we intend to base a McEliece's like cryptosystem on such class of codes.

However, as it is already the case for the distinguisher fo classical Goppa codes \cite{MT21}, it seems complicated to turn this distinguisher into an efficiant structural attack. But, having an algebraic explanation of the structure of the square of the dual of one--point Goppa--like codes is still desirable if we want to perform an attack using square code considerations. It also provides an rigorous way to choose initial parameters to secure our cryptosystem.

We hope that this work will lead to several proposal/improvements on SSAG code based McElice cryptosystem in the future.

\TODO{Organisation}
The paper is organized as follows. In Section \ref{sec:preli}, we recall basic definitions about linear codes, AG codes and $C_{a,b}$ curves.


\section{Preliminaries}\label{sec:preli}

\subsection{Linear codes, subfield subcode and trace code}

%\TODO{citations needed}

In this section, we briefly introduce some notation and basic definitions for linear codes, subfield subcodes, and trace codes. Furthermore, we present significant results that employ component-wise product and trace map. For further details about linear codes, we refer the reader to \cite{MS86}.

\noindent Let $\fq$ be the finite field with $q$ elements, $m > 0$ and denote  by $\fqm$ the finite extension of $\fq$ of degree $m$. A linear code $\calC$ over $\fqm$ is a subspace of $\fqm^n$. The integer $n$ is called its length and we denote by $k$ its dimension, and $d$ its minimum distance. We say that $\calC$ is a $[n,k,d]_{q^m}$ code or that it has parameters $[n,k,d]_{q^m}$. Given a linear code of length $n$, its dual code $\calC^{\perp}$ is defined by 
\[\calC^{\perp}=\left\lbrace x \in \fqm^n \mid c \cdot x=0, \text{ for all } c \in \calC \right\rbrace,\]  
where $\cdot$ denotes the usual scalar product. It is easily verified that is $\calC$ is a $[n,k]_{q^m}$ code, then $\calC^{\perp}$ is a $[n,n-k]_{q^m}$ code.
Given any linear code with lenght $n$ and dimension $k$, its rate is defined by the ration $R := \frac{k}{n}$.

\noindent The Schur product of two vectors $\mathbf{a}$,$\mathbf{b} \in \fqm^n$ is defined as 
\[ \mathbf{a} \star \mathbf{b} := (a_1b_1,\cdots,a_nb_n). \]
This definition can be extended to codes the following way: If $\calC$ and $\calD$ over $\fqm$ are two codes over $\fqm$ with same length $n$, then their Schur product is defined as
\[ \calC \star \calD := \Span{\mathbf{c} \star \mathbf{d} \mid \mathbf{c} \in \calC, \mathbf{d} \in \calD}. \]
Moreover, if $\calC = \calD$, we call $\calC^{\star 2} := \calC \star \calC$ the square of $\calC$. The following lemma gives a first estimation of the dimension of Schur product of codes.

\begin{lemma}[{\cite[Proposition 10]{MT21}}] \label{lem:known_bounds}
Let $\calC$ and $\calD$ be two linear codes over $\fqm$ with same length $n$ and respective dimension $k_{\calC}$ and $k_{\calD}$. We have
\begin{enumerate}
	\item $\dim_{\fqm}(\calC \star \calD) \leq \min\{k_{\calC}k_{\calD},n\}$;
	\item If $\calC$ is sufficiently random, we have
\[ \dim_{\mathbb{F}_{q^m}}(\calC^{\star2}) \leq \mathrm{min}\left(n,\binom{k_{\calC}+1}{2}\right) . \]
Especially, if $\calC^{\star2}$ does not fill the full space, we expect to have 
	\[ \dim_{\fqm}(\calC^{\star2}) = \binom{k_{\calC}+1}{2}.\]
	\end{enumerate}
\end{lemma}

Given a linear code $\calC$ over $\fqm$, there exists two constructions of codes on the subfield $\fq$, namely its subfield subcode and its trace code. The \emph{subfield subcode} of $\calC$, denoted by $\calC|_{\fq}$ is the linear code over $\fq$ defined by 
\[\calC|_{\fq}=\calC \cap \mathbb{F}_q^n.\]
Again, if $\calC$ is a $[n,k,d]_{q^m}$ code, then $\calC\mid_{\fq}$ is a $[n,\geq n-m(n-k),\geq d]_q$ code.

Let $\operatorname{Tr}_{\mathbb{F}_{q^m}/\fq}$ be the trace operator on the field $\mathbb{F}_{q^m}$ with respect to $\mathbb{F}_q$, that is defined by
\[\Tr{x} = x + x^q + ... + x^{q^{m-1}},\]
for any $x \in \fqm$. We can extend this definition to a vector $\mathbf{c} \in \fqm^n$ by $$\Tr{\mathbf{c}}= (\Tr{c_1},\cdots,\Tr{c_n}).$$ 

\noindent Given a linear code $\calC$ of length $n$ and dimension $k$ over $\fqm$, its \emph{trace code} over $\fq$ is the image under the trace operator $\Tr{\calC}$. It is a linear code of length $n$ over $\fqm$, which dimension satisfies
\begin{equation}\label{eq:dim_trace}
\dim_{\mathbb{F}_q} \Tr{\calC} \leq \min\{m\dim_{\fqm} \calC,n\}.
\end{equation}

Subfield subcodes and trace codes are related by the duality operator, as stated by Delsarte's theorem.

\begin{thm}[Delsarte's theorem \cite{Del75}] \label{th:delsarte}
Let $\calC$ be a linear code over $\fqm$. Then
\[\left(\calC \cap \fq^n\right) = \Tr{\calC^{\perp}}.\]
\end{thm}

\subsection{AG and SSAG codes} \label{section:AG_codes}

\paragraph{Definitions.} Let $\calX$ be a smooth and irreducible projective curve over $\fqm$ of genus $\mathfrak{g}$. A \emph{divisor} on $\calX$ over $\fqm$ is a formal sum of places over $\fqm$ $D=\sum \nu_P(D) P$ where $\nu_P(D)$ are integers which are all zero except for a finite number of places $P$. We denote by $\Div(\calX)$ the set of $\fqm$--divisors on $\calX$ (we omit the dependence on $\fqm$).

For a divisor $D \in \Div(\calX)$, we define its \emph{support} $\Supp(D)$ as the finite set of places $P$ such that $\nu_P(D)$ is non--zero and its \emph{degree} $\deg D$ as $\deg D=\sum \nu_P(D) \deg(P)$. We say that a divisor $D \in \Div(\calX)$ is \emph{effective} if for all $P \in \Supp(D)$, we have $\nu_P(D) \geq 0$, in which case we write $D \geq 0$.
A non--zero function $f \in \fqm(\calX)$ defines a divisor, called \emph{principal}, denoted by $(f)=\sum \nu_P(f) P$. The \emph{Riemann-Roch space} of $D$ is defined as the $\fqm$ vector space
$$ \calL(D) := \set{f \in \fqm(\calX) \mid (f) + D \geq 0} \cup \set{0},$$
of dimension $\ell(D)$.
Let also $\calP \subseteq \fqm(\calX)$ be a set of $n$ distinct rational points such that $\Supp(D) \cap \calP = \varnothing$.
We can consider the AG code 
$$\calC := \calC_{\calL}(\calX,\calP,D) := \set{\left(f(P)\right)_{P \in \calP} \mid f \in \calL(D)},$$
which is a length $n$ linear code over $\fqm$, with dimension $k \leq \ell(D)$. If $n \geq \deg D$, then its dimension is exactly equal to $\ell(D)$ and its minimum distance is bounded from below by $n-\deg D$.\\
From Riemann-Roch theorem, we have
$$ \ell(D) \geq \deg(D) +1 - \mathfrak{g},$$ 
with equality if $\deg(D) \geq 2\mathfrak{g}-1$.\\
In the case of AG codes, we have a special notation for the subfield subcode, namely
\[\ssag{\calX,\calP,D} := C_{\calL}(\calX,\calP,D)|_{\fq}.\]

The dual of an AG code can be described as a residue code (see \cite{Sti09} for more details), \emph{i.e.}
$$ C_{\calL}(\calX,\calP,D)^{\perp} = C_{\omega}(\calX,\calP,D).$$ 
Residue and evaluation codes are connected by the following result.

\begin{proposition} [{\cite[Proposition 2.2.10]{Sti09}}] \label{prop:dual_AG_codes}
Let $C_{\calL}(\calX,\calP,D)$ be an AG-code defined on a curve $\calX$. Then 
\[C_{\calL}(\calX,\calP,D)^{\perp} = C_{\calL}(\calX,\calP,D^{\perp}),\]
with $D^{\perp} = D_{\calP}-D+W$, where $D_{\calP} = \sum\limits_{P \in \calP} P$ and $W=(\omega)$ is a canonical divisor such that for all $P \in \calP$, $\nu_P(\omega)=-1$ and $\mathrm{Res}_{\omega}(P)=1$. 
\end{proposition}


\paragraph{First estimation of the dimension of the square of the trace of an AG code.}

In this paper, we aim to bound the dimension of the square of the dual of a SSAG codes. From Delsarte's theorem, this is equivalent to study the square of the trace of some structured AG code. 
This fact is possible thanks to a result from \cite{MT21}, valid for any linear code:

\begin{proposition}[{\cite[Proposition 15]{MT21}}] \label{prop:Tr_BoundSchurSquare}
	Let $\calC$ be a linear code over $\fqm$. Then we have 
	\begin{equation} \label{eq:key_equation} \Tr{\calC}^{\star2} := ((\calC|^{\perp}_{_{{\mathbb{F}_q}}})^{\perp})^{\star2} \subseteq \sum\limits_{i=0}^{\lfloor{m/2} \rfloor} \Tr{\calC\star \calC^{q^i}},
	\end{equation}
\end{proposition}

From this result, we can deduce an upper bound for the dimension of the square of the dual of any linear code.

\begin{coro} [{\cite[Corollary 16]{MT21}}]\label{coro:first_bound_square_of_trace}
	Let $\calC$ be any $\fqm$-linear code. Then 
	\begin{equation} \label{eq:mumford_bound}
		\dim_{\fq}\Tr{\calC}^{\star2} \leq m \cdot \dim_{\fqm}(\calC^{\star 2}) + \binom{m}{2} (\dim_{\fqm}(\calC))^2.
	\end{equation}
	Furthermore, if $\dim_{\fq} \Tr{\calC} = m \cdot \dim_{\fqm}(\calC)$, then 
	\[\dim_{\fq} \Tr{\calC}^{\star2} - \binom{\dim_{\fq} \Tr{\calC}+1}{2} \leq m \cdot \dim_{\fqm} \calC^{\star 2} - \binom{\dim_{\fqm} (\calC)+1}{2}.\]
\end{coro}

The above corollary implies that if the dimension of a square code is smaller than we expect from a random code, namely
\[ \dim_{\fqm} (\calC^{\star 2}) < \binom{\dim_{\fqm} (\calC)+1}{2},\]
then this property survives for the trace code, \emph{i.e.}
\[\dim_{\fq} \Tr{\calC}^{\star 2} < \binom{\dim_{\fq} \Tr{\calC}+1}{2}.\]

In particular, this is the case for Reed-Solomon codes (see \cite{MT21}, Proposition 11) and more generally for AG codes under some lower bound condition on the degree of its divisor.


\begin{proposition} [{\cite[Theorem 6]{Mum70}}] \label{prop:mumford_result}
	Let $F,G$ be two divisors of $\calX$ such that $\deg(G) \geq 2\mathfrak{g}+1$ and $\deg(F) \geq 2\mathfrak{g}$. Then
	\[ \calL(F) \cdot \calL(G) = \calL(F+G),\]
	where $\calL(F) \cdot \calL(G) := \Span{ f \cdot g : f,g \in \calL(F) \times \calL(G)}$.
\end{proposition}

\noindent As a consequence, for an AG code  $C_{\calL}(\calX,\mathcal{P},D)$ with $\deg(D) \geq 2\mathfrak{g}+1$, then 
\[ C_{\calL}(\calX,\mathcal{P},D)^{\star2} = C_{\calL}(\calX,\calP,2D).\]
The Riemann-Roch theorem thus gives
\[ \dim_{\fqm}(C_{\calL}(\calX,\mathcal{P},D)^{\star2}) = 2\deg(D)+1-\mathfrak{g}= \deg(D) + \dim_{\fqm}(C_{\calL}(\calX,\mathcal{P},D)), \]
which is much smaller than the expected dimension given in Lemma \ref{lem:known_bounds} 2. (thus providing a distinguisher for AG-codes). Combined with \eqref{eq:key_equation}, we get a first upper bound on the dimension of the square of the trace of an AG code.


\begin{coro} \label{coro:1st_bound_mumford}
	Let $\mathcal{C} := C_{\calL}(\calX,\mathcal{P},D)$ be a $k$--dimensional AG-code on $\calX$ associated to a degree $s \geq 2g(\calX)+1$ divisor. Then
	\[ \dim_{\fq}\Tr{\calC}^{\star2} := \dim_{\fq} (\ssag{\calX,\calP,D^{\perp}}^{\perp})^{\star2}  \leq \binom{mk+1}{2} - \dfrac{m}{2} (k(k-1)-2s).\]
\end{coro}


\begin{proof}
	From Proposition \ref{prop:mumford_result}, we have $\dim_{\fqm}(\calC)^{\star2} = 2s+1-g = k+s$, since we have inequality in the Riemann-Roch theorem. Thus, \eqref{eq:mumford_bound} leads to
	\begin{align*}
		\dim_{\fq}\Tr{\calC}^{\star2} &\leq m(k+s) + \binom{m}{2}k^2 \\
		&\leq (2k+2s+mk^2-k^2) \dfrac{m}{2} \\
		&\leq (k(mk+1)-k^2+k+2s) \dfrac{m}{2} \\
		&\leq \binom{mk+1}{2} - \dfrac{m}{2}(k(k-1)-2s) .
	\end{align*}
\end{proof}

The above corollary says that due to the algebraic structure of AG codes, the dimension of the square of the dual of a SSAG code is smaller than the expected value for random codes, which is $\binom{mk+1}{2}$. However, this bound is far from being optimal, and our aim for the rest of the paper is to improve it in some specific cases. 





\subsection{$C_{a,b}$ curves} \label{section:C_a,b_codes}

Throughout this paper, we will be dealing with algebraic geometry codes defined over a $C_{a,b}$ curve. This section is dedicated to some properties on this well--studied class of curves. For further details, we refer to \cite{Miu93}. 


\begin{definition} \label{def:C_ab_curves} 
Let $a,b$ be coprime positive integers. A $C_{a,b}$ curve over $\fqm$ is a curve $\calX_{a,b}$ having an irreducible, affine and non singular plane model with equation

\begin{equation} \label{eq:equation_C_ab}
f_{a,b}(x,y) = \alpha_{0a}y^a + \alpha_{b0}x^b + \sum \alpha_{ij}x^iy^j = 0,
\end{equation}
where $f_{a,b} \in \fqm[X,Y]$ and the sum is taken over all couples $(i,j) \in \set{0,\cdots,b} \times \set{0,\cdots,a}$ such that $ai+bj < ab$.
\end{definition}

Any curve $\calX_{a,b}$ defined by an equation as in \eqref{eq:equation_C_ab} has a unique point at infinity, denoted by $P_{\infty}$. Moreover, as a plane curve, its genus is given by
\[\mathfrak{g}_{a,b}:=\dfrac{(a-1)(b-1)}{2}.\]



%\noindent Later, we will be interested in AG codes with a specific choice of divisor, which are defined below.

\paragraph{The point at infinity.} We will consider codes obtained by evaluating functions on $\calX_{a,b}$ which are regular everywhere, except maybe at the unique point at infinity $P_\infty$ on $\calX_{a,b}$. These functions then belong to the ring 
\begin{equation}\label{eq:O_Pinf}
\mathcal{O}_{P_\infty}=\bigcup_{s \geq 0} \calL(s P_\infty)
\end{equation} %
where each Riemann--Roch space $\calL(s P_\infty)$ has an explicit basis as follows:

\begin{equation} \label{eq:basis_L(sP_inf)}
    \calL(sP_{\infty}) = \Span{x^iy^j \mid 0 \leq i, 0\leq j\leq a-1 \ \mathrm{and} \ ai+bj \leq s}.
\end{equation}
%
In summary, any function that is regular on all $\calX_{a,b}$ except maybe at $P_\infty$ is a polynomial in the functions $x$ and $y$. 

\begin{definition}[Weighted degree]
Given a monomial of the form $x^iy^j \in \calO_{P_\infty}$, , we define its weighted degree by
\[ \degab{x^iy^j} := ai+bj.\]
%
Any function $f \in \calO_{P_\infty}$ can be written of the form $$f = x^{\beta}y^{\alpha} + f'(x,y),$$
with $\alpha \leq a-1$ and $f' \in \calL(sP_{\infty})$ such that any monomial $x^iy^j$ of $f'$ satisfies $ai+bj < \degab{x^{\beta}y^{\alpha}}$ and $j \leq a-1$. We can then define the leading term of $f$ as $\mathrm{LT}(f) := x^{\beta}y^{\alpha},$ and extend the definition of weighted degree to any such function by setting 
\[\degab{f} := \degab{\mathrm{LT}(f)}.\]
\end{definition}

It is easy to check that for any function $f \in \calO_{P_\infty}$, its weighted degree $\degab{f}$ is equal to the smallest integer $s$ such that $f$ belongs to the Riemann--Roch space $\calL(sP_{\infty})$.
%
This way, any function in $\calL(sP_\infty)$ can be seen as a polynomial in $x$ and $y$ such that $\degab{f}\leq s$. 

\begin{remark}
It is worth noting that $\calO_{P_\infty}$ is a valuation ring, with valuation $v_{P_\infty}$. Then for every $f \in \calO_{P_\infty}$, we have $\degab{f}=-v_{P_\infty}(f)$. We prefer handling the weighted degree rather than the valuation due to its similarities with the degree of univariate 
polynomials. We will notably perform division, using Gr\"obner bases, and, as expected in the univariate case, the degree of the remainder is 
\textit{generally} smaller than the dividend's one.
\end{remark}

\section{Goppa--like AG codes}

\subsection{The codes}

Let $D$ be an effective divisor of positive degree $s$ on a smooth and irreducible projective curve $\calX$ over $\fqm$. Take a rational function $g \in \fqm(\calX)$ such that $g \notin \calL(D)$. Given a set of points $\calP \in \calX(\fqm)$ such that $\calP \cap \Supp(g) = \varnothing$ and $\calP \cap \Supp(D) = \varnothing$, we consider the AG code
\[\calC := \calC_{\calL}(\calX,\calP,D+(g))=\set{\left(f(P)g(P)^{-1}\right)_{P \in \calP} \mid f \in \calL(D)}.\]

\begin{definition} \label{def:Goppa--like_AG_code}
The Goppa--like AG code associated to $\calC$ is defined as the dual of its subfield subcode, \emph{i.e.}
$$ \Gamma(\calP,D,g) := \calC^{\perp}|_{\fq}.$$
\end{definition}

\paragraph{Why the terminology \textit{Goppa--like}?} The terminology Goppa--like is motivated by the fact that here, the rational function $g$ plays the role of the Goppa polynomial. Goppa codes are nothing but Goppa--like AG codes from the projective line $\calX=\PP^1$.

Let $r$ be an integer. The Generalized Reed--Solomon (GRS) code of degree $r$, of support $\mathbf{x} \in \fqm^n$ and multiplier $\mathbf{y} \in (\fqm^*)^n$ is defined as

\[\GRS_r(\mathbf{x},\mathbf{y})=\{(y_1f(x_1),y_2f(x_2),\dots,y_nf(x_n)), f \in \fqm[X] \text{ such that } \deg f < r \}.\]

Take a univariate polynomial $g$ of degree $r$ such that $g(x_i) \neq 0$ for every  $i \in \{1,\dots,n\}$. The the Goppa code of order $r$ and support $\mathbf{x} \in \fqm^n$ is defined as
\[\Gamma_r(\mathbf{x},g)= \GRS_r(\mathbf{x},\mathbf{y})^\perp|_{\fq}\]
where $\mathbf{y}=(g(x_1)^{-1},g(x_2)^{-1},\dots,g(x_n)^{-1})$.

Represent the $\fqm$--points of $\PP^1$ by the couples $\PP^1(\fqm)=\{[1:x] \mid x \in \fqm\} \cup \{P_\infty\}$ for $P_\infty=[0:1]$. Take $\calP=\{[1:x_1],[1:x_2],\dots,[1:x_n]\}$ and $D=(r-1)P_\infty$. Finally, the polynomial $g$ can be seen as a function on $\PP^1$ which lies in $L(rP_\infty)$ but not in $L((r-1)P_\infty)$. Then both constructions match: $\Gamma_r(\mathbf{x},g)=\Gamma(\calP,D,g)$.


\paragraph{Relation with Cartier codes.} Cartier codes \cite{Cou14} are also defined as a geometric generalisation of Goppa codes, since well-known properties on Goppa codes naturally extend to them.

The link with Goppa--like AG codes is the following: by definition, a Cartier code is a subcode of the subfield subcode of a residue code (see \cite[Proposition 4.3]{Cou14}), which actually means that for the good choice of divisor, a Cartier code is a subcode of the corresponding Goppa--like AG code. Moreover, \cite[Theorem 5.1]{Cou14} provides a sufficient condition for both constructions to be equal. More precisely, let us consider a Goppa--like AG code $\Gamma(\calP,D,g)$  and set $G := D+(g)$. Then the cartier code $\mathrm{Car}_q(\calP,G)$ (see \cite[Definition 4.2]{Cou14}) satisfies $\mathrm{Car}_q(\calP,G) \subseteq \Gamma(\calP,D,g)$, and 

$$ \dim_{\fq} \left( \Gamma(\calP,D,g)/ \mathrm{Car}_q(\calP,G)\right) \leq m \cdot i(G_1),$$
where $G_1$ is any divisor such that 
\begin{equation} \label{eq:divisor_G_1}
G \geq qG_1 \ \mathrm{and} \ G \geq G_1.
\end{equation}
 Above, $i(G_1)$ stands for the index of speciality of $G_1$ (see \cite[Definition 1.6.10]{Sti09}), and we have $\deg(G_1) > 2\mathfrak{g}-2 \Rightarrow i(G_1) =0.$ The biggest divisor (with respect to the degree) satisfying \eqref{eq:divisor_G_1} is given by 
 $$\left\lfloor \frac{G}{q} \right\rfloor := \sum\limits_{P \in \Supp(G^+)} \left\lfloor\frac{\nu_P(G^+)}{q}\right\rfloor ¨P + \sum\limits_{P \in \Supp(G^-)}\nu_P(G^-)P,$$
 where $G^+ = D+(g)_0$ and $G^-=(g)_\infty$ are such that $G=G^+-G^-$.
Thus, whenever $\deg\left\lfloor \dfrac{G}{q} \right\rfloor > 2\mathfrak{g}-2$, the corresponding Cartier code coincides with the Goppa--like AG code. 


\subsection{On the dimension of the square of the dual of a Goppa--like code} \label{section:1st_improvement}

In this section, we aim to generalize the properties found by the authors of \cite{MT21} in Section 6, in the context of Goppa--like AG code. Let us consider the AG code 
$$\calC := \calC_{\calL}(\calX,\calP,D+(g))$$
as defined in Section \ref{def:Goppa--like_AG_code}. Applying \eqref{eq:key_equation} yields   
\begin{equation} \label{eq:key_equation_Goppa--like} 
\Tr{\calC}^{\star 2} = (\Gamma(\calP,D,g)^{\perp})^{\star2} \subseteq \sum\limits_{i=0}^{\lfloor{m/2} \rfloor} \Tr{\calC\star \calC^{q^i}}.
\end{equation}



Below, we discuss how to improve the upper bound given in Corollary \ref{coro:1st_bound_mumford}, which is valid for all subfield subcodes of AG codes. The idea is to use the specific algebraic structure of our code inherited from the choice of its divisor.

\noindent In fact, notice that the code $\calC$ is monomially equivalent to $C_{\calL}(\calX,\calP,D)$. 
More precisely, since $\calL(D+(g)) = g^{-1} \cdot \calL(D),$
we can easily see that
\begin{equation} \label{eq:equiv_divisors}
\dim_{\fqm} \calL(D+(g))\cdot \calL(D+(g))^{q^i} = \dim_{\fqm} \calL(D)\cdot \calL(D)^{q^i},
\end{equation}
where $\calL(D+(g))\cdot \calL(D+(g))^{q^i}$ corresponds to the evaluation space of the product code $\calC \star \calC^{q^i}$. We can use the fact that $D$ is effective to estimate the dimension of the terms appearing in the sum of \eqref{eq:key_equation_Goppa--like}:


\begin{lemma} \label{lem:bound_dim_Tr(C*C^q^i)}
  Suppose $s \geq 2\mathfrak{g}-1$. Then for all $i \geq 0$, we have 
   $$\dim_{\fq} \Tr{\calC\star \calC^{q^i}} \leq m\left(s\left(q^i+1\right)+1-\mathfrak{g}\right).$$
\end{lemma}

\begin{proof}
For $i \geq 0$ and $f_1,f_2 \in \calL(D)$, we have 
$$(f_1f_2^{q^i}) = (f_1)+q^i(f_2) \geq -D +(-q^iD) = -(q^i+1)D,$$
which proves the inclusion
$$ \calL(D)\cdot \calL(D)^{q^i} \subseteq \calL((q^i+1)D).$$
Since $C_{\calL}(\calX,\calP,(q^i+1)D) = \set{\mathrm{Ev}_{\calP}(f) : f \in \calL((q^i+1)D)}$, we deduce 
$$ \dim_{\fqm} \calL(D) \cdot \calL(D)^{q^i} \leq \dim_{\fqm}\left(C_{\calL}(\calX,\calP,(q^i+1)D)\right) = s(q^i+1)+1-\mathfrak{g},$$
the last equality coming from the Riemann-Roch theorem (since $s \geq 2\mathfrak{g}-1$). The result follows from \eqref{eq:equiv_divisors} and the usual upper bound on the dimension of the trace of a code. 
\end{proof}

\noindent This simple Lemma yields an upper bound on the dimension of the square of the dual of Goppa--like codes.

\begin{proposition} \label{prop:bound_dim_using_inclusions}
    Let $\calC := \calC_{\calL}(\calX,\calP,D+(g))$ be as above, and suppose $s \geq 2\mathfrak{g}-1$. Set $k := \dim_{\fqm}\calC = s+1-\mathfrak{g}$ and  $e := \min\left(\left\lfloor \frac{m}{2} \right\rfloor,\left\lfloor \log_q\left(\frac{k^2}{s}\right)\right\rfloor\right)$. Then
    $$\dim_{\fq} (\Gamma(\calP,D,g)^{\perp})^{\star 2} \leq \binom{mk+1}{2} - \dfrac{m}{2}\left(k(k-1)(2e+1)-2s\left(\dfrac{q^{e+1}-1}{q-1}\right)\right).$$
\end{proposition}

\begin{proof}
    From Proposition \ref{prop:Tr_BoundSchurSquare} and Lemma \ref{lem:bound_dim_Tr(C*C^q^i)}, we have 
    \begin{align*}
        \dim_{\fq}(\Gamma(\calP,D,g)^{\perp})^{\star 2}
        & \leq \sum\limits_{i=0}^{\lfloor m/2 \rfloor} \dim_{\fq} \Tr{\calC \star \calC^{q^i}} \\
        & \leq \sum\limits_{i=0}^{e} m(s(q^i+1)+1-\mathfrak{g})  + \sum\limits_{i=e+1}^{\lfloor m/2 \rfloor} \Tr{\calC \star \calC^{q^i}} \\
        & \leq \sum\limits_{i=0}^{e} m(sq^i+k) + \left( \frac{m-1}{2} -e \right)mk^2 \\
        & \leq \frac{m}{2}\left(2k(e+1)+2s\left(\dfrac{q^{e+1}-1}{q-1}\right)+k^2(m-1)-2ek^2  \right) \\
        & \leq \binom{mk+1}{2} -  \dfrac{m}{2}\left(k(k-1)(2e+1)-2s\left(\dfrac{q^{e+1}-1}{q-1}\right)\right).
    \end{align*}
Here, the inequality holds for any $e \in \set{0,\dots,\lfloor \frac{m}{2} \rfloor}$. To get the best bound, we maximize the expression $$ \dfrac{m}{2}\left(k(k-1)(2e+1)-2s\left(\dfrac{q^{e+1}-1}{q-1}\right)\right)$$ with respect to $e$. Removing the constant parts, this is equivalent to find the maximum of the function
$$T(e) = ek^2-s\dfrac{q^{e+1}}{q-1}$$
over $\set{0,\dots,\lfloor \frac{m}{2} \rfloor}$ in the discrete domain of non-negative integers.  
We compute the discrete derivative:
\begin{align*}
    \Delta T(e) = T(e+1)-T(e) &= (e+1)k^2-s\dfrac{q^{e+2}}{q-1} - \left(ek^2-s\dfrac{q^{e+1}}{q-1}\right) \\
                              &= k^2 - sq^{e+1}.
\end{align*}
This function is decreasing with $e$, and the smallest value for which $\Delta T(e) \leq 0$ corresponds to its maximum. It is the smallest value of $e$ such that $k^2 \leq sq^{e+1}$, \emph{i.e.}
$$e =  \left\lfloor \log_q\left(\dfrac{k^2}{s}\right)\right\rfloor.$$
\end{proof}

\paragraph{Why imposing $g \notin \calL(D)$?} In the definition of Goppa--like AG code, we ask for the function $g$ to lie outside the Riemann--space $\calL(D)$. By doing so, we cannot see all alternant codes as Goppa--like AG codes on $\calX=\PP^1$. However, to make these codes resistant to a distinguished based on the square of their dual, it is crucial to make sure that the dimension of $\Tr{\calC}^{\star 2}$ is not abnormally small compared to the typical value given in Corollary \ref{coro:first_bound_square_of_trace}. 

First, forcing $g \notin \calL(D)$ ensures that the unit vector $(1,\dots,1)$ do no lie in the AG code $\calC$. Otherwise, $(1,\dots,1)$ belongs to $\calC^{q^i}$ for every $i \in \{0,\dots, \lfloor{m/2} \rfloor\}$ and each term in the sum on the right hand-side would contain a copy $\Tr{\calC}$.

More precisely, if the function $g$ lied in $\calL(D)$, then the divisor $D+(g)$ would be effective and for every $i \geq 0$, we would have the inclusion $\calL((q^i+1)D) \subset \calL((q^{i+1}+1)D)$. Therefore, in the proof of Proposition \ref{prop:bound_dim_using_inclusions}, when bounding from above the dimension of the sum by the sum of the dimensions of the trace codes, we would have no chance to get a sharp bound.

\section{One--point Goppa--like AG code on $C_{a,b}$-curves}

The bound given in section \ref{section:1st_improvement} can be impreoved by considering more structured codes, \emph{i.e.} one--point Goppa--like AG codes on $C_{a,b}$ curves.

\subsection{Definition}


Below, we define a specific class of Goppa--like AG codes on a $C_{a,b}$ curve, associated to a divisor which is equivalent to the one--point divisor $sP_\infty$. 

\noindent Throughout the rest of the paper, we fix a $C_{a,b}$ curve $\calX_{a,b}$ as defined in Definition \ref{def:C_ab_curves}.

\begin{definition} \label{def:one--point_Goppa--like_AG_codes_on_C_a,b_curves}
Let $s'>s$ be two integers such that there exists a function $g \in \calL(s'P_\infty)$ with $\degab{g}=s'$. Given a set of points  $\calP \subset \calX_{a,b}(\F_{q^m})$ such that $\calP \cap \Supp(g) = \varnothing$, we define the one--point Goppa--like AG code associated to $\calP,s$ and $g$ as 
\[\Gamma(\calP,sP_\infty,g) := \calC_{\calL}(\calX_{a,b},\calP,(sP_\infty+(g))^{\perp})|_{\fq}.\]
\end{definition}

This definition might be restrictive, but it is reasonable as these codes can be encoded quickly thanks to the nice basis of $\calL(sP_\infty)$ (see \eqref{eq:basis_L(sP_inf)}), which is desirable if we aim to build a McEliece cryptosystem based on this family of codes. Moreover, this property will be key in the upcoming sections as it allows a better understanding of the square of the dual of any one--point Goppa--like AG code, under some condition on $s$ and $s'$. \\


\noindent In the next sections, we generalize the result given in \cite{MT21} in the case of classical Goppa codes, by defining a weighted euclidean division on the ring $\calO_{_P\infty}$ (see Eq \eqref{eq:O_Pinf}), whose elements are seen as bivariate polynomials. 

\subsection{Weighted euclidean division}

\noindent The following proposition generalizes the classical euclidean division of polynomial in the case of function in the ring $\calO_{_P\infty}$ (see Eq \eqref{eq:O_Pinf}) with respect to the weighted degree $\deg_{a,b}$.


\begin{proposition}\label{prop:div_grob}
Let $m$ be a positive integer and let $g \in \calO_{_P\infty}$. Write $g=x^\beta y^\alpha +g'(x,y)$ with $\alpha < a$ and $g' \in \calO_{_P\infty}$ such that $\degab{x^\beta y^\alpha}>\degab{g'}$.
For any function $f \in \calO_{P_\infty}$, one can write $f=f_1g+f_2$ with 
\[f_2 \in \calR(g):= \Span{x^u y^v \mid u \leq \beta + b-1 \text{ and } v\leq a-1 \text{ not both }  u \geq \beta \text{ and } v \geq \alpha}.\]
Moreover, we had $\degab{f_2} \leq \degab{f}$ and $\dim_{\fqm} \calR(g) = \degab{g}=s'.$ 
\end{proposition}

\begin{proof}
		Since $f \in \calO_{P_\infty}$, one can see $f$ as a bivariate polynomial of $x$ and $y$ (see Equation \ref{eq:O_Pinf}). In the polynomial ring $\F_{q^m}[x,y]$, we perform the division of $f$ by a Grobner basis of the ideal generated by the equation $\phi_{a,b}$ of the curve $\calX_{a,b}$ and the polynomial $g$ with respect to the monomial order $\prec$ defined as follows : $x^uy^v \prec x^{u'}y^{v'}$ if
	\[ \degab{x^uy^v} < \degab{x^{u'}y^{v'}} \text{ or } \left(\degab{x^uy^v} = \degab{x^{u'}y^{v'}}  \text{ and } u < u'\right).\]
	With respect to this order, the leading term og $g$ is $\LT{g}=x^\beta y^\alpha$. The fact that $f_2$ lies in $\calR(g)$ and the result on the dimension of $\calR(g)$ both follow from \cite[Proposition 4]{GH00}.
	
	Finally, if we had $\degab{f} < \degab{f_2}$ with $f=f_1 g +f_2$, this would mean that $\LT{f_2}=-\LT{f_1 g}=\lambda x^uy^v$ for some $\lambda \in \F_{q^m}^*$ with both $u \geq \beta$ and $v \geq \alpha$, which is not possible by definition of $\calR\left(g\right)$.
\end{proof}


\begin{lemma} \label{lem:weighted_division}
	Let $i \geq 1$ and $s'>s \geq 0$. Let $g \in \calL\left(s'P_\infty\right)$ and $f \in \calL\left(\left(s'(q^i+1)-1\right)P_\infty\right)$. Then there exists $f' \in \calR\left(g^{q^i-q^{i-1}+1}\right)$ such that $\Tr{\frac{f}{g^{q^i+1}}} = \Tr{\frac{f'}{g^{q^i+1}}} $.
\end{lemma}

\begin{proof}

By Proposition \ref{prop:div_grob}, we can write $f=f_1 g^{q^i-q^{i-1}+1} +f_2$ with
	$f_2 \in \calR\left(g^{q^i-q^{i-1}+1}\right)$ with $\degab{f_2} \leq \degab{f}$. Therefore
	\[\Tr{\frac{f}{g^{q^i+1}}}=\Tr{\frac{f_1 g^{q^i-q^{i-1}+1}}{g^{q^i+1}}} +\Tr{\frac{f_2}{g^{q^i+1}}}= \Tr{\frac{f_1^qg}{g^{q^i+1}}} +\Tr{\frac{f_2}{g^{q^i+1}}}. \]
	
	By definition, the second term has the expected form. Let us examinate the first term. If $f_1=0$, we are done. Otherwise, the definition of $f_1$ gives
$\degab{f_1} =\degab{f} - s'(q^i-q^{i-1}+1)$, and
	\begin{align*}
	\degab{f_1^qg} 	&= q \degab{f_1} + s'\\
					&= q\degab{f} - s'(q-1)(q^i+1).
	\end{align*} 
Then  $\degab{f_1^qg} < \degab{f}$ if and only if $\degab{f} < s'(q^i+1)$, which holds by definition of $f$. Repeating the division process on $f_1^qg$, as the weighted degree decreases, we can find a function $f' \in \calR\left(g^{q^i-q^{i-1}+1}\right)$ such that $\Tr{\frac{f}{g^{q^i+1}}} = \Tr{\frac{f'}{g^{q^i+1}}} $.
\end{proof}

\begin{definition} \label{def:T_i's}
For any $1 \leq i \leq \lfloor\frac{m}{2}\rfloor$, we define
$$\calT_i(s,g)= \Tr{g^{-(q^i+1)}\cdot \left( \calR\left(g^{q^i-q^{i-1}+1}\right)  \cap \calL(s(q^i+1)P_\infty)\right)}$$
and we set $$\calT_0(s,g) := \Tr{g^{-2} \cdot \calL(2sP_\infty}).$$
\end{definition}


\noindent Let $i \geq 0$ and $f \in \calL(sP_\infty) \cdot \calL(sP_\infty)^{q^i} \subseteq \calL(s(q^i+1)P_\infty)$. Then 
Lemma \ref{lem:weighted_division} entails that 
$$\Tr{\dfrac{f}{q^{i+1}}} \in \calT_i(s,g).$$
Thus, for all $i \in \set{0,\dots,\lfloor \frac{m}{2} \rfloor}$, we have \begin{equation} \label{eq:Tr(C*C^q^i)_dans_T_i}
\Tr{\calC \star \calC^{q^i}} \subseteq \calT_i(s,g).
\end{equation}

\noindent The above inclusion can be used to improve the bound given in Proposition \ref{prop:bound_dim_using_inclusions}, provided that we can compute efficiently the dimension of $\calT_i(s,g)'s$, which is the purpose of the upcoming section.

%--------Last changes----------%

\subsection{Upper bound in Goppa--like case}

In the proposition below, we study the intersection 
\begin{equation}\label{eq:def_Mi}
M_i(s,g):=R\left(g^{q^{i}-q^{i-1}+1}\right) \cap \calL(s(q^i+1)P_\infty)
\end{equation}
for every $i \in \set{1,\dots,\lfloor m/2 \rfloor}$ in order to better grasp the trace codes $\calT_i(s,g)$'s introduced in Definition \ref{def:T_i's}.

First let us set some notation. Fix $i \in \set{1,\dots,\lfloor m/2 \rfloor}$. Write $\LT{g}=x^\beta y^\alpha$ with $a\beta + b\alpha=s'$. By reducing modulo the equation $\phi_{a,b}$ of the curve $\calX_{a,b}$, we can write $g^{q^i-q^{i-1}+1}$ such that its leading term with respect to the monomial order $\prec$ is
\begin{equation}\label{eq:gi}
	\LT{g^{q^i-q^{i-1}+1}}=x^{\beta_i} y^{\alpha_i}
\end{equation}
 where $\alpha_i \in \set{0,\dots,a-1}$ is the remainder of the Euclidean division of $\alpha (q^i-q^{i-1}+1)$ by $a$ and 
 \begin{equation}\label{eq:value_beta_prime}
 	\beta_i=\beta(q^i-q^{i-1}+1) + b \, \frac{\alpha(q^i-q^{i-1}+1)-\alpha_i}{a}= \frac{s'(q^i-q^{i-1}+1)-b\alpha_i}{a}.
 \end{equation}
%
Depending on the weighted degree $s'$ of $g$, we can compute the exact dimension of $M_i(s,g)$ defined in Equation \eqref{eq:def_Mi}.

\begin{proposition} \label{prop:dim_M_i's}
%Let $i \in \set{1,\dots,\lfloor m/2 \rfloor}$. Set $\alpha_i \in \set{0,\dots,a-1}$ to be the remainder of the Euclidean division of $\alpha (q^i-q^{i-1}+1)$ by $a$ and set $\beta_i = \dfrac{s'(q^i-q^{i-1}+1)-b\alpha_i}{a}$.   
    \begin{enumerate}
        \item If $s'(q^i-q^{i-1}+1) > s(q^i+1)+a$, then $M_i(s,g) = \calL(s(q^i+1)P_\infty)$ ;
        
        \item If $s'(q^i-q^{i-1}+1) \leq s(q^i+1)+1-2\mathfrak{g}_{a,b}$, then $M_i(s,g) = R(g^{q^{i}-q^{i-1}+1})$;
        
        \item If there exists $v^* \in \set{1,\dots,\alpha_i-1}$ such that
        $$ s(q^i+1)+a-b(a+v^*-\alpha_i) < s'(q^i-q^{i-1}+1) \leq  s(q^i+1)+a-b(a+v^*-1-\alpha_i),$$
        we have 
          $$\dim_{\fqm}(M_i(s,g)) = \sum\limits_{v=v^*}^{a-1} \left\lfloor \dfrac{s(q^i+1)-bv}{a} \right\rfloor + v^*(\beta_i+b) + a-v^*.$$
        
        \item Else, there exists $v^* \in \set{\alpha_i+1,\dots,a}$ such that
        $$ s(q^i+1)+a-b(v^*-\alpha_i) < s'(q^i-q^{i-1}+1) \leq  s(q^i+1)+a-b(v^*-1-\alpha_i),$$
        in which case 
        $$\dim_{\fqm}(M_i(s,g)) = \sum\limits_{v=v^*}^{a-1} \left\lfloor \dfrac{s(q^i+1)-bv}{a} \right\rfloor + v^*\beta_i + \alpha_ib +a-v^*.$$
    \end{enumerate}
\end{proposition}

\begin{proof}


%By assumption, we have $\mathrm{LT}(g)=x^{\beta}y^{\alpha}$, with $\alpha \leq a-1$ and $\degab{g}=a\beta +b\alpha=s'$. 
%Thus, $\mathrm{LT}(g^{q^i-q^{i-1}+1})=x^{\beta (q^i-q^{i-1}+1)}y^{\alpha (q^i-q^{i-1}+1)}$. Since $\alpha (q^i-q^{i-1}+1)$ might be bigger that $a-1$, we perform the following euclidean division:
%$$\alpha (q^i-q^{i-1}+1) = \delta a + \alpha_i, \ 0 \leq \alpha_i \leq a-1.$$
%Using the equation of the curve, we can then suppose that $\mathrm{LT}(g^{q^i-q^{i-1}+1})=x^{\beta_i}y^{\alpha '}$, with 
%$$\beta_i := \beta (q^i-q^{i-1}+1) + \left\lfloor \frac{\alpha (q^i-q^{i-1}+1)-\alpha_i}{a} \right\rfloor \ \mathrm{and} \ \alpha_i \in \set{0,\dots,a-1}.$$
%Since $\degab{g^{q^i-q^{i-1}+1}} = s'(q^i-q^{i-1}+1),$ we also have 
%\begin{equation} \label{eq:value_beta_prime}
%\beta_i = \dfrac{s'(q^i-q^{i-1}+1)-b\alpha_i}{a}.
%\end{equation}
Using the notation above, we can write
\begin{align*}
\calR\left(g^{q^i-q^{i-1}+1}\right) &:= \mathrm{Span}_{\fqm} \left\{x^uy^v \mid u \leq \beta_i+b-1 , v \leq a-1 \ \mathrm{not \ both} \ u \geq \beta_i \ \mathrm{and} \ v \geq \alpha_i\right\} \\
&= \mathrm{Span}_{\fqm}    \left\{ \begin{array}{c}
         1,x,\dots,x^{\beta_i +b-1},   \\
         \cdots \\
         y^{\alpha_i -1},y^{\alpha_i -1}x,\dots,y^{\alpha_i -1}x^{\beta_i +b-1}, \\
          y^{\alpha_i},y^{\alpha_i}x,\dots,y^{\alpha_i}x^{\beta_i-1}, \\
         \cdots \\
         y^{a-1},y^{a-1}x,\dots,y^{a-1}x^{\beta_i-1}
    \end{array}
    \right\}
\end{align*}
Next, we define for any $v \in \set{0,\dots,a-1}$:
$$\ell^i_v := \max \set{u \geq 0 \mid x^uy^v \in \calL(s(q^i+1)P_\infty)} = \left\lfloor \dfrac{s(q^i+1)-bv}{a}\right\rfloor,$$
implying
\begin{equation*}
\calL(s(q^i+1)P_\infty) = \mathrm{Span}_{\fqm}    \left\{ \begin{array}{c}
         1,x,\dots,x^{\ell^i_0},   \\
         y,yx,\dots,yx^{\ell^i_1}, \\
         \cdots \\
         y^{a-1},y^{a-1}x,\dots,y^{a-1}x^{\ell^i_{a-1}}
    \end{array}
    \right\}.
\end{equation*}
With these notations, we have a description of a basis of both spaces $R(g^{q^i-q^{i-1}+1})$ and $\calL(s(q^i+1)P_\infty)$, leading to an exact formula for the dimension of their intersection $M_i(s,g)$ for any value of $i$:

\begin{equation} \label{eq:exact_dimension_M_i}
\dim_{\fqm} M_i(s,g) = \sum\limits_{v=0}^{\alpha_i-1} \min(\beta_i+b,\ell_v^i+1) + \sum\limits_{v=\alpha_i}^{a-1} \min(\beta_i,\ell_v^i+1).
\end{equation}

It remains to compute the corresponding minima with respect to $v$:
\begin{itemize}
    \item[(i)] If $0 \leq v \leq \alpha_i-1$, by using \eqref{eq:value_beta_prime}, we get
    \begin{align*}
        \beta_i+b \leq \ell_v^i +1 \iff s'(q^i-q^{i-1}+1) \leq F(v) := s(q^i+1)+a-b(a+v-\alpha_i).
    \end{align*}
    \item[(ii)] Otherwise, $\alpha_i \leq v \leq a-1$ and
    \begin{align*}
        \beta_i \leq \ell_v^i +1 \iff s'(q^i-q^{i-1}+1) \leq G(v) := s(q^i+1)+a-b(v-\alpha_i).
    \end{align*}
\end{itemize}
Note that both $F$ and $G$ are decreasing with $v$, and we easily check that $F(0) = G(a)$. Thus, we have the following sequence of integers
$$F(\alpha_i-1) \leq \dots \leq F(0) = G(a) \leq G(a-1) \leq \dots \leq G(\alpha_i).$$
Depending on the value of $s'$, there is a few cases  to consider:
\begin{itemize}
    \item $s'(q^i-q^{i-1}+1) >G(\alpha_i)$, in which case $M_i(s,g) = \calL(s(q^i+1)P_\infty)$;
    \item $s'(q^i-q^{i-1}+1) \leq F(\alpha_i-1)$, and $M_i(s,g) = \calR(g^{q^i-q^{i-1}+1})$;
    \item There exists $v^* \in \set{1,...,\alpha_i-1}$ such that $F(v^*) < s'(q^i-q^{i-1}+1) \leq F(v^*-1)$;
    \item There exists $v^* \in \set{\alpha_i,...,a}$ such that $G(v^*) < s'(q^i-q^{i-1}+1) \leq G(v^*-1)$.

\end{itemize}
The formulas for the dimension of $M_i(s,g)$ follows from the above computations and \eqref{eq:exact_dimension_M_i}.
\end{proof}

Remark that case $\textit{1)}$ corresponds to the case where $\calT_i(s,g) = \Tr{\calL(s(q^i+1)P_\infty)}, $ which will produce the same bound as the one given in Proposition \ref{prop:bound_dim_using_inclusions}.
Instead, we will focus on case $2)$, since in this case, we can show that thesequence $(\calT_i(s,g))_i$ is increasing for the inclusion.





\begin{proposition} \label{prop:inclusion_T_i's} Let $i^* \in \set{0,\dots,\lfloor\frac{m}{2}\rfloor-1}$ be the smallest integer such that 
$$sq^{i^*} \geq (s'-s)(q^{i^*+1}-q^{i^{*}}+1)+2\mathfrak{g}_{a,b}-1.$$ 
Then
$$\calT_{i^*}(s,g) \subseteq \calT_{i^*+1}(s,g) \subseteq \dots \subseteq \calT_{\lfloor \frac{m}{2}\rfloor}(s,g).$$
\end{proposition}

\begin{proof}
From Proposition \ref{prop:dim_M_i's}, \textit{2)}, we know that the condition on $s$ and $s'$ implies $$M_{i^*+1}(s,g) = R(g^{q^{i^*+1}-q^{i^*+1-1}+1}).$$ 
Since the function $$i \mapsto \dfrac{s(q^i+1)+1-2\mathfrak{g}_{a,b}}{q^i-q^{i-1}+1}$$ 
is increasing with $i$, we also have
\begin{equation} \label{eq:structure_M_i}
M_{i}(s,g) = R(g^{q^{i}-q^{i-1}+1}), \ \forall \ i \in \set{i^*,\dots,\left\lfloor \frac{m}{2}\right\rfloor+1}.
\end{equation} 

We now prove the inclusions of the $\calT_i's$, supposing first that $i^* \neq 0$ (since the definition of $\calT_0$ is a bit different). Let $i \in \set{i^*,\dots,\lfloor \frac{m}{2} \rfloor}$, and note that 
$$\calT_i(s,g) := \set{\Tr{\frac{f}{g^{q^i+1}}} \mid f \in \calR(g^{q^i-q^{i-1}+1})\cap \calL(s(q^i+1)P_\infty)}.$$
Given $\Tr{\frac{f}{g^{q^i+1}}}$ in $\calT_i(s,g)$, we want to show that it belongs to $\calT_{i+1}(s,g)$. Applying Proposition \ref{prop:div_grob} by replacing $f$ with $fg^{q^{i+1}-q^i}$ and $g$ by $g^{q^{i+1}-q^i+1}$, we obtain
\begin{equation} \label{eq:division_i}
fg^{q^{i+1}-q^i} = f_1g^{q^{i+1}-q^i+1} + f_2,
\end{equation}
with $f_2 \in \calR(g^{q^{i+1}-q^i+1}) = M_i(s,g)$ (using \eqref{eq:structure_M_i}) and $\degab{f_2} \leq \degab{fg^{q^{i+1}-q^i}}$. Next, we write
\begin{align*}
    \Tr{\frac{f}{g^{q^i+1}}} &= \Tr{\frac{fg^{q^{i+1}-q^i}}{g^{q^{i+1}+1}}} \\
                             &= \Tr{\frac{f_1g^{q^{i+1}-q^i+1}}{g^{q^{i+1}+1}}} + \Tr{\frac{f_2}{g^{q^{i+1}+1}}} \\
                             &= \Tr{\frac{f_1^qg}{g^{q^{i+1}+1}}} + \Tr{\frac{f_2}{g^{q^{i+1}+1}}}.
\end{align*}
By assumption, we immediately have that $\Tr{\frac{f_2}{g^{q^{i+1}+1}}} \in \calT_{i+1}(s,g).$

If $f_1=0$, we are done. Otherwise, we have from \eqref{eq:division_i}:
$$\degab{f_1} = \degab{fg^{q^{i+1}-q^i}} - \degab{g^{q^{i+1}-q^i+1}} = \degab{f}-s'.$$
Thus
\begin{align*}
     \degab{f_1^qg} < \degab{fg^{q^{i+1}-q^i}} & \iff q\degab{f} +(1-q)s' < \degab{f} +s'(q^{i+1}-q^i) \\
%                                               & \iff (q-1)\degab{f} < s'(q^{i+1}-q^i+q-1) \\
                                               & \iff \degab{f} < s'(q^{i}+1),
\end{align*}
which is true since in particular $f \in \calL(s(q^i+1)P_\infty)$ and $s<s'$. We can thus repeat the division process until eventually we obtain a quotient $f_1$ equal to zero. This shows that $\calT_i(s,g) \subseteq \calT_{i+1}(s,g)$.

In the case $i^*=0$, we can repeat the same process exepted that we also have to prove that $\calT_0(s,g) \subseteq \calT_1(s,g)$, which differs from the others cases due to the definition of $\calT_0$. We treat this case separately: let $\Tr{\frac{f}{g^2}} \in \calT_0(s,g)$, for some $f \in \calL(2sP_\infty)$. Using Proposition \ref{prop:div_grob}, this time replacing $f$ with $fg^{q-1}$ and $g$ with $g^{q+1}$, we have
$$fg^{q-1} = f_1g^q + f_2,$$ with $f_2 \in \calR(g^q) = M_1(s,g)$ (using \eqref{eq:structure_M_i}).
Thus, we can write
    $$ \Tr{\frac{f}{g^2}} = \Tr{\frac{f_1^qg}{g^{q+1}}}  + \Tr{\frac{f_2}{g^{q+1}}}, $$
with $\Tr{\frac{f_2}{g^{q+1}}} \in \calT_1(s,g)$. Since $\degab{f_1} = \degab{fg^{q-1}} - \degab{g^q} = \degab{f}-s'$, we have 
\begin{align*}
     \degab{f_1^qg} < \degab{fg^{q-1}} & \iff q\degab{f} +(1-q)s' < \degab{f} + s'(q-1)\\
                                               & \iff (q-1)\degab{f} < 2s'(q-1)\\
                                               & \iff \degab{f} < 2s',
\end{align*}
which holds since $s<s'$ and $f \in \calL(2sP_\infty)$. Repeating the division process until we found a quotient equal to zero shows that $\calT_0(s,g) \subseteq \calT_1(s,g)$. The others cases work as the case $i^* \geq 1$.
\end{proof}

Combining \eqref{eq:Tr(C*C^q^i)_dans_T_i} with both the above propositions lead to a better understanding of the dimension of the square of the dual of one--point Goppa--like AG codes. 

\begin{coro} \label{coro:folklore_upper_bound} \mathieu{bon je suis pas fan de ce à quoi ça ressemble :(} \\
With notation as in Proposition \ref{prop:inclusion_T_i's} and $k:=\dim_{\fqm}C_{\calL}(\calX_{a,b},\calP,sP_\infty+(g))$, the dimension of $\Gamma(\calP,sP_\infty,g)^{\perp})^{\star 2}$ is bounded from above by
\begin{align*}
\min  \left\{ \begin{array}{ll}
\left(\frac{m-1}{2}-e\right)mk^2+\dim_{\fq}\left(\sum\limits_{i=0}^e \calT_i(s,g) \right), \ \mathrm{if} \ e \geq i^* \\
\left(\frac{m-1}{2}-e\right)mk^2 + ms'(q^e-q^{e-1}+1) + \dim_{\fq}\left(\sum\limits_{i=0}^{i^*-1} \calT_i(s,g) - \dim_{\fq} \calT_e(s,g) \cap   \sum\limits_{i=0}^{i^*-1} \calT_i(s,g)\right)\  \mathrm{overwise},
\end{array} 
\right.
\end{align*}
where the minimum is taken over all $e \in \set{0,\dots,\lfloor \frac{m}{2} \rfloor}$.
\end{coro}

\begin{proof}
From \eqref{eq:key_equation} and under the assumption on $s$ and $s'$, we have
        \begin{align*}
        \dim_{\fq} (\Gamma_s(\calP,g)^{\perp})^{\star 2}
        & \leq \sum\limits_{i=0}^{\lfloor m/2 \rfloor} \dim_{\fq} \Tr{\calC \star                   		\calC^{q^i}} \\
        & \leq \dim_{\fq} \sum\limits_{i=0}^{e}\calT_i(s,g) + \sum\limits_{i=e+1}^{\lfloor m/2 \rfloor} \Tr{\calC \star \calC^{q^i}} \\
                & \leq \dim_{\fq} \sum\limits_{i=0}^{e}\calT_i(s,g) + \left( \frac{m-1}{2} -e \right)mk^2,
        \end{align*}
 for all $e \in \set{0,\dots,\lfloor \frac{m}{2} \rfloor}$. We then distinguish two cases depending on the relative position of $i^*$ with respect to $e$ and the fact that 
 $$ \sum\limits_{i=i^*}^e \calT_i(s,g) = \calT_e(s,g),$$
 using Proposition \ref{prop:inclusion_T_i's}. The final bound is optained by taking the smallest value with respect to $e$.  
\end{proof}

Despite the fact that the upper bound given in Corollary \ref{coro:folklore_upper_bound} can be explicitely computed with the knowledge of $s$ and $s'$, it is hard to give an explicit formula for any parameters, since the dimension of the sum of $\calT_i(s,g)$'s is had to manipulate. Howerver, if we suppose that $i^*=0$, we can precise the above result. \TODO{explain why this case is the most intersting ?}

\begin{thm} \label{thm:bound_with_T_i's_inclusion}
Suppose $s \geq (s'-s)q+2\mathfrak{g}_{a,b}-1$ and let $e^* := \min\left(\left\lfloor \frac{m}{2} \right\rfloor, \left\lceil \log_q\left(\frac{k^2}{s'(q-1)^2}\right)\right\rceil+1\right)$. Then
$$\dim_{\fq} (\Gamma_s(\calP,g)^{\perp})^{\star 2}\leq \binom{mk+1}{2} - \dfrac{m}{2}(k^2(2e^*+1)+k-2s'(q^{e^*}-q^{e^*-1}+1)). $$
\end{thm}

\begin{proof}
The condition $s \geq (s'-s)q+2\mathfrak{g}_{a,b}-1$ exactly implies that $i^*=0$ and $$\calT_0(s,g) \subseteq \calT_1(s,g) \subseteq \dots \subseteq \calT_{\lfloor \frac{m}{2}\rfloor}(s,g),$$ 
see Proposition \ref{prop:inclusion_T_i's}. Thus, using Corollary \ref{coro:folklore_upper_bound}, we get

\begin{align*}
        \dim_{\fq} (\Gamma_s(\calP,g)^{\perp})^{\star 2}
        & \leq \min \left(ms'(q^e-q^{e-1}+1) + \left( \frac{m-1}{2} -e \right)mk^2 \right)\\
        & \leq \min \left(\frac{m}{2}\left(2s'(q^e-q^{e-1}+1)+k^2(m-1)-2k^2e  \right)\right) \\
        & \leq \min\left(\binom{mk+1}{2} - \dfrac{m}{2}\left(k^2(2e+1)+k-2s'(q^e-q^{e-1}+1)\right)\right).
\end{align*}

the minimum being taking over $e \in \set{1,\dots,\lfloor \frac{m}{2} \rfloor}$. We also used $\dim_{\fq}\calT_e(s,g) \leq m \dim_{\fqm} R(g^{q^e-q^{e-1}+1})$.
To get the best bound, we need to maximize the function
$$T(e) = ek^2-s'(q^e-q^{e-1}+1)$$
over $\set{1,\dots,\lfloor \frac{m}{2} \rfloor}$.
We compute the discrete derivative:
\begin{align*}
    \Delta T(e) = T(e+1)-T(e) &= (e+1)k^2- s'(q^{e+1}-q^e+1) - ek^2 + s'(q^e-q^{e-1}+1) \\
                              &= k^2 - s'q^{e-1}(q-1)^2.
\end{align*}
This function is decreasing with $e$, and the smallest value for which $\Delta T(e) \leq 0$ corresponds to its maximum. It is the smallest value of $e$ such that $k^2 \leq s'q^{e-1}(q-1)^2$, \emph{i.e.}
$$e =  \left\lceil \log_q\left(\dfrac{k^2}{s'(q-1)^2}\right)\right\rceil+1.$$
\end{proof}

\noindent Several computational experiments showed then when the code $\calC:=\calC_{\calL}(\calX_{a,b},\calP,sP_\infty+(g))$ is sufficiently random (\emph{i.e} the rational function $g$ the square of the dual of the corresponding one--point Goppa--like AG code given in Theorem \ref{thm:bound_with_T_i's_inclusion} is actually sharp, leading to  distinguisher if the parameters of $\calC$ are not well-chosen. \\

In the last section, we will discuss how to chose efficiently the parameters of a one--point Goppa--like AG code in order to resist this distinguisher.

\section{Analysis of the distinguisher}

\noindent In the previous section, we provided an (experimentally) sharp upper bound on the dimension of the square of the dual of a Goppa--like code, which could leads to a distinguisher for the corresponding code. 

\noindent More precisely, let $\calC := C_\calL(\calX_{a,b},\calP,D)$ be an AG code as above, \emph{i.e.} with $D = sP_\infty +(g)$ and $\degab{g} = s'>s\geq 2\mathfrak{g}_{a,b}-1$. We showed that if $s$ and $s'$ are such that $s \geq (s'-s)q+2\mathfrak{g}_{a,b}-1$, then
\begin{equation} \label{eq:best_upper_bound}
\dim_{\fq} (\Gamma_s(\calP,g)^{\perp})^{\star 2} \leq \min \left(\frac{m}{2}\left(2s'(q^{e^*}-q^{e^*-1}+1)+k^2(m-1-2e^*)  \right),n\right),
\end{equation}
where $e^* := \left\lceil \log_q\left(\dfrac{k^2}{s'(q-1)^2}\right)\right\rceil+1$. Thus, the code is distinguishable from a random one if the bound above is less than the lenght $n$ of the code. We can easily check when this case occurs, by starting to bound from above the maximal length possible: since $\calP \cap \Supp(g) = \varnothing,$ this maximum is reached when $\calP = \calX_{a,b}(\fqm) \backslash \Supp(g)$ and $g$ has only one zero, that is
$$n = \# \calP = |\calX_{a,b}(\fqm)| \leq q^m-1+2\mathfrak{g}_{a,b}\sqrt{q^m},$$
using the Hasse-Weil bound. In order to secure the corresponding Goppa--like code against our attack, the parameters have to be chosen such that 
\begin{equation} \label{eq:cond_not_to_distinguish}
\frac{m}{2}\left(2s'(q^{e^*}-q^{e^*-1}+1)+k^2(m-1-2e^*)  \right)\geq q^m-1+2\mathfrak{g}_{a,b}\sqrt{q^m}.
\end{equation}

\noindent In what follows, we will focus on two specific classes of $C_{a,b}$--curves: First, we will compute the maximal (with respect to the dimension) codes we can distinguish in the case where $\calX_{a,b}$ is an elliptic curve. This case is relevant since it is the closest to the case of classical Goppa codes, and we will see that our results are very similar to the one given in \cite{MT21}. Next up, we will focus on the particular case of the Hermitian curve, which also turns out to be a $\calX_{a,b}$ curve. It is well-known to be a good candidate to construct efficient codes as it is a maximal curve. Moreover, we will show that due to the high genus of the curve, any Goppa--like code defined on it can not be distinguished with our attack, since our upper bound is always bigger than the maximal possible lenght.


\subsection{High rate distinguishable codes in the case of elliptic curves}


In this section, we focus on the case where $\calX_{a,b}$ is an elliptic curve, meaning that $a=2$ and $b=3$. For some set of parameters which produces codes of cryptographic size, we then compute the maximal distinguishable value of $s$. In order to get closer to the case of classical Goppa codes, we also fix $s'=s+1$.

\begin{table}[h]
\begin{center}
\begin{tabular}{|c|c|c||c|c|c|c|c|c|}
    \hline
    $q$ & $m$ & $n$ & Largest distinguishable $s$ & Corresponding rate\\
    \hline \hline
     $2$ & $12$ & $4218$ & $14$ & $0,963$ \\
    \hline 
     $2$ & $13$ & $6688$ & $18$ & $0,982$  \\
    \hline \hline
     $3$ & $7$ & $2186$ & $15$ & $0,962$ \\
    \hline
     $3$ & $8$ & $6393$ & $24$ & $0,977$ \\
    \hline \hline
     $5$ & $5$ & $3043$ & $27$ & $0,961$  \\
    \hline
     $5$ & $6$ & $4500$ & $22$ & $0,971$ \\
    \hline
     $5$  & $6$ & $6688$ & $30$ & $0,976$ \\
    \hline \hline
     $7$ & $4$ & $2395$ & $27$ & $0,957$ \\
    \hline
      $7$ & $5$ & $4650$ & $26$ & $0,971$ \\
    \hline
      $7$ & $5$ & $8192$ & $37$ & $0,979$ \\
    \hline \hline
      $17$ & $3$ & $4820$ & $92$ & $0,943$ \\
    \hline
\end{tabular}
\caption{Largest distinguishable Goppa--like AG code in elliptic case}
\end{center}
\end{table}

We can see that the rate of the distinguishable codes are roughly the same as the one given in \cite{MT21} in the case of classical Goppa codes, \emph{i.e} we are only able to distinguish high rate Goppa--like codes.



\subsection{Codes on the Hermitian curve}

In this section, we investigate Hermitian Goppa--like codes, since the Hermitian curve is a particular case of $C_{a,b}$ curve. This case is interesting since Hermitian codes are good candidates to reach the best length for a fixed base field. The good news is that we can prove that Hermitian Goppa--like codes resist our distinguisher, since we can show that \eqref{eq:cond_not_to_distinguish} always holds in the Hermitian settings, essentially because  the genus of the curve is too high. 

\noindent Let us first recall some known results about the Hermitian curve (see for example \cite{Sti09}). Let $m \geq 1$  be an even integer and denote by $q_0 := q^{m/2}$, so that $\fqm = \fqo$. The Hermitian curve $\calH$ over $\fqo$ is defined by the equation
$$\calH : y^{q_0}+y = x^{q_0+1}.$$
Its genus is given by $\mathfrak{g}_{\calH} = \frac{q_0(q_0-1)}{2}$ and it is a maximal curve, \emph{i.e.} $\#\calH(\fqo) = q_0^3+1$.


\begin{proposition} \label{prop:Hermitian_Goppa_like_are_secured}
    Suppose $s \geq (s'-s)q+2\mathfrak{g}_{a,b}-1$. Then for any choice of $g$ and $\calP$, the Goppa--like code $\Gamma_s(\calP,g)$ resists to the distinguisher given above.
\end{proposition}

\mathieu{Je suis pas super satisfait de la preuve, il y a surement plus simple...}

\TODO{Raccourcir ces calculs + test sur des $s$ plus petits}

\begin{proof}
    As discussed at the beginning of the section, the code cannot be distinguished if the upper bound for the dimension of the square of its dual given in Theorem \ref{thm:bound_with_T_i's_inclusion} is larger than the length of the code. In the Hermitian case, we know exactly the number of rational points, and thus the length $n$ of the Goppa--like code is at most $q_0^3-1$. Since $m$ is even, we are left to prove that 
    \begin{equation} \label{eq:dont_distinguish_Hermitian_case}
    \mathfrak{B}(e^*) := ms'(q^{e^*}-q^{e^*-1}+1) + \left( \frac{m}{2}-e^*\right)mk^2 \geq q_0^3-1,
    \end{equation}
    where $k := s+1-\mathfrak{g}_{\calH}$ stands  for the dimension of the corresponding AG-code.
    \begin{itemize}
        \item [-] If $e^* < \frac{m}{2}$, then $\mathfrak{B}(e^*) > mk^2$. With the condition on $s$, we know that $s \geq 2\mathfrak{g}_{\calH}+q-1$. This give $k \geq \mathfrak{g}_{\calH}+q$ and thus
        \begin{align*}
\mathfrak{B}(e^*) - (q_0^3-1) 
&> m(\mathfrak{g}_{\calH}^2+2\mathfrak{g}_{\calH}q+q^2)-(q_0^3-1)\\
& > \frac{m}{4}(q_0^4-2q_0^3+q_0^2) + mq(q_0^2-q_0+q) -q_0^3+1 \\
& \geq \frac{1}{2}(q_0^4-2q_0^3+q_0^2)+4(q_0^2-q_0+2)-q_0^3+1 \quad (\mathrm{since} \ m\geq 2 \ \mathrm{and} \ q\geq 2)\\
& > \frac{q_0}{2} (q_0^3-4q_0^2+9q_0-8) > 0,
        \end{align*}
        since $q_0 \geq 2$. Inequality \eqref{eq:dont_distinguish_Hermitian_case} holds in this case.
        \item[-] If $e^* = \frac{m}{2},$ then since $q_0=q^{m/2}$, we have $\mathfrak{B}\left(\frac{m}{2}\right) = ms'(q_0-q_0q^{-1}+1)$. Moreover, $s'>s$ implies $s' \geq 2\mathfrak{g}_{\calH}+q$. This gives
        \begin{align*}
           \mathfrak{B}\left(\frac{m}{2}\right) - (q_0^3-1) 
           &\geq m(2\mathfrak{g}_{\calH}+q)(q_0-q_0q^{-1}+1)-q_0^3+1 \\ 
           &\geq 2\left(q_0^2(q_0-1)\left(\frac{q-1}{q}\right)+q_0(q_0-1)+q_0(q-1)+q\right)-q_0^3+1 \\
           & \geq q_0^3\left(2\left(\frac{q-1}{q}\right)-1\right) + 2q_0^2\left(1-\left(\frac{q-1}{q}\right)\right) + 2(q_0(q-2)+q)+1.
        \end{align*}
        Clearly, the last expression is minimal for $q=2$, so we finally gets
        $$\mathfrak{B}\left(\frac{m}{2}\right) \geq q_0^2 + 5 >0,$$
        which proves \eqref{eq:dont_distinguish_Hermitian_case} in this case and conclude the proof.
    \end{itemize}
\end{proof}

Consequently, it is still reasonable to consider the Hermitian curve to build efficient SSAG code-based cryptosystem.


\clearpage
\bibliography{biblio}
\bibliographystyle{alpha}



\end{document}
